\chapter[Tool demonstration: Molten Salt Breeder Reactor]{SaltProc 
demonstration: Molten Salt Breeder Reactor}

This chapter describes the fuel cycle analysis of the \gls{MSBR} obtained 
using the open-source python package, SaltProc. The development was initially 
started as a part of my master thesis \cite{rykhlevskii_advanced_2018} in 
2017.  In this effort, for verification purpose, I assumed ideal extraction 
efficiency (e.g., 100\% of target isotope mass extracted) because all 
available in the literature results also rely on this assumption.

The main results presented in this chapter have been published in: A. 
Rykhlevskii, J.W. Bae, and K. D. Huff, ``Modeling and simulation of online 
reprocessing in the thorium-fueled molten salt breeder reactor,'' 
\textit{Annals of Nuclear Energy}, 128 (2019): 366--379. The high-fidelity, 
full-core \gls{MSBR} model has been presented at the 2017 \gls{ANS} Winter 
Meeting in Washington D.C. The fuel salt composition evolution has been 
presented at the 2018 Blue Waters Symposium in Sunriver, OR. The obtained 
results relevant to \gls{MSBR} analysis have been compared against those 
obtained by Benjamin R. Betzler and colleagues for simplified unit cell model, 
adopting the in-house code ChemTriton. 


\section{Introduction}
The thorium-fueled \gls{MSBR} was developed in the early 1970s by \gls{ORNL} 
specifically to explore the promise of the thorium fuel cycle, which uses 
natural thorium instead of enriched uranium. With continuous fuel 
reprocessing, the \gls{MSBR} realizes the advantages of the thorium fuel cycle 
because the $^{233}$U bred from $^{232}$Th is almost instantly\footnote{\space 
The fertile $^{232}$Th is transmuted into the $^{233}$Th after capturing a 
neutron. Next, this isotope decays to the $^{233}$Pa ($\tau_{1/2}$=21.83m), 
which finally decays to the $^{233}$U ($\tau_{1/2}$=26.967d).} recycled back 
into the core  \cite{betzler_modeling_2016}. The chosen fuel salt, 
LiF-BeF$_2$-ThF$_4$-UF$_4$, has a melting point of $499^\circ$C, a low vapor 
pressure at operating temperatures, and good flow and heat transfer properties 
\cite{robertson_conceptual_1971}. 

In this work, we analyzed the \gls{MSBR} neutronics and fuel cycle
to 
establish its equilibrium core composition. Additionally, we
compared 
predicted operational and safety parameters of the
\gls{MSBR} at both the 
initial and equilibrium states to characterize
the evolution of its safety 
case over time. Moreover, these depletion simulations
determined the 
appropriate $^{232}$Th feed rate for maintaining criticality and enabled 
analysis of the overall \gls{MSBR} fuel cycle
performance. Finally, benefits 
of online fission product removal in the thermal spectrum \gls{MSBR} were 
identified.


\section{Molten Salt Breeder Reactor design and model description}
The \gls{MSBR} vessel has a diameter of 680 cm and a height of 610 cm. It 
contains a molten fluoride fuel-salt mixture that generates heat in the active 
core region and transports that heat to the primary heat exchanger by way of 
the primary salt pump. In the active core region, the fuel salt flows through 
channels in moderating and reflecting graphite blocks. Fuel salt at 
565$^{\circ}$C enters the central manifold at the bottom via four  
40.64-cm-diameter nozzles and flows upward through channels in the lower 
plenum graphite. The fuel salt exits at the top at about 704$^{\circ}$C 
through four equally spaced nozzles which connect to the salt-suction pipes 
leading to primary circulation pumps. The fuel salt drain lines connect to the 
bottom of the reactor vessel inlet manifold.

Figure~\ref{fig:serpent_plan_view} shows the configuration of the \gls{MSBR} 
vessel, including the ``fission" (zone I) and ``breeding" (zone II) regions 
inside the vessel. The core has two radial zones bounded by a solid  
cylindrical graphite reflector and the vessel wall. The central zone, zone I, 
in which 13\% of the volume is fuel salt and 87\% graphite, is composed of 
1,320 graphite cells, 2 graphite control rods, and 2 safety\footnote{ These 
rods needed for emergency shutdown only.} rods. The under-moderated zone, zone 
II, with 37\% fuel salt, and radial reflector, surrounds the zone I core 
region and serves to diminish neutron leakage. Zones I and II are surrounded 
radially and axially by fuel salt (figure~\ref{fig:serpent_zoneII}). This 
space for fuel is necessary for injection and flow of molten salt.
\begin{figure}[t] % replace 't' with 'b' to \centering
	\includegraphics[width=\textwidth]{ch3/view_serpent.png}
	\caption{$XY$ (left) and $XZ$ (right) views of Serpent \gls{MSBR} model 
	(figure reproduced from Rykhlevskii \emph{et al.} 
	\cite{rykhlevskii_modeling_2019}).}
	\label{fig:serpent_plan_view}
\end{figure}

\begin{figure}[t!] % replace 't' with 'b' to \centering
	\includegraphics[width=\textwidth]{ch3/ser_zone_II.png}
	\caption{Detailed view of \gls{MSBR} two zone model. 
		Yellow represents fuel salt, purple represents graphite, and aqua 
		represents the reactor vessel. Figure reproduced from Rykhlevskii 
		\emph{et al.} \cite{rykhlevskii_modeling_2019}.}
	\label{fig:serpent_zoneII}
\end{figure}

Since reactor graphite experiences significant dimensional changes due to 
neutron irradiation, the reactor core was designed for periodic replacement. 
Based on the experimental irradiation data from the \gls{MSRE}, the core 
graphite lifetime is about 4 years and the reflector graphite lifetime is 30 
years \cite{robertson_conceptual_1971}.

There are eight symmetric graphite slabs with a width of 15.24 cm in zone II, 
one of which is illustrated in Figure~\ref{fig:serpent_zoneII}. The holes in 
the centers are for the core lifting rods used during the core replacement 
operations. These holes also allow a portion of the fuel salt to flow to the 
top of the vessel for cooling the top head and axial reflector.  
Figure~\ref{fig:serpent_zoneII} also shows the 5.08-cm-wide annular 
space between the removable core graphite in zone II-B and the permanently 
mounted reflector graphite. This annulus consists entirely of fuel salt, 
provides space for moving the core assembly, helps compensate for the 
elliptical dimensions of the reactor vessel, and serves to reduce the damaging 
flux at the surface of the graphite reflector blocks. 

$^{135}$Xe is a strong neutron poison, and some fraction of this gas is 
absorbed by graphite during \gls{MSBR} operation. ORNL calculations show 
that for unsealed commercial graphite with helium permeability 10$^{-5}$ 
cm$^2$/s the calculated poison fraction is less than 2\%  
\cite{robertson_conceptual_1971}.  This parameter can be improved by using 
experimental graphites or by applying sealing technology. The effect of the 
gradual poisoning of the core graphite with xenon is not treated here.

\subsection{Core zone I}
The central region of the core, called zone I, is made up of graphite 
elements, each $10.16$cm$\times$ 10.16cm$\times$396.24cm. Zone I has 4 
channels for control rods: two for graphite rods which both regulate and shim 
during normal operation, and two for backup safety rods consisting of boron 
carbide clad to assure sufficient negative reactivity for emergency situations.

These graphite elements have a mostly rectangular shape with lengthwise ridges 
at each corner that leave space for salt flow elements. Various element sizes 
reduce the peak damage flux and power density in the center of the core to 
prevent local graphite damage.  Figure~\ref{fig:I_element_ref} shows the 
elevation and plan views of graphite elements of zone I 
\cite{robertson_conceptual_1971} and their Serpent model 
\cite{rykhlevskii_full-core_2017}.
\begin{figure}[ht!] % replace 't' with 'b' to \centering
	\includegraphics[width=\textwidth]{ch3/zone_I_element_ref.png}
	\caption{Graphite moderator elements for zone I : reference design (left)
		\cite{robertson_conceptual_1971} and Serpent model (right) 
		\cite{rykhlevskii_full-core_2017}.  Yellow 
		represents fuel salt, purple represents graphite, and aqua represents 
		the reactor vessel. Figure reproduced from Rykhlevskii \emph{et al.} 
		\cite{rykhlevskii_modeling_2019}.}
	\label{fig:I_element_ref}
\end{figure}

\subsection{Core zone II}
Zone II, which is undermoderated, surrounds zone I. Combined with the bounding 
radial reflector, zone II serves to diminish neutron leakage. Two kinds of 
elements form this zone: large-diameter fuel channels (zone II-A) and 
radial graphite slats (zone II-B). 

Zone II has 37\% fuel salt by volume and each element has a fuel channel 
diameter of 6.604cm. The graphite elements for zone II-A are prismatic with
elliptical dowels running axially between the prisms. These dowels
isolate the fuel salt flow in zone I from that in zone II.  
Figure~\ref{fig:II_element_ref} shows the shapes and dimensions of these 
graphite elements and their Serpent model. Zone II-B elements are rectangular 
slats spaced far enough apart to provide the 0.37 fuel salt volume fraction. 
The reactor zone II-B graphite 5.08cm-thick slats vary in the radial dimension 
(average width is 26.67cm) as shown in figure~\ref{fig:serpent_zoneII}. Zone 
II serves as a blanket to achieve the best performance: a high breeding ratio 
and a low fissile inventory. The harder neutron energy spectrum in zone II 
enhances the rate of thorium resonance capture relative to the fission rate, 
thus limiting the neutron flux in the outer core zone and reducing the neutron 
leakage \cite{robertson_conceptual_1971}. 

The sophisticated, irregular shapes of the fuel elements challenge an accurate 
representation of zone II-B. The suggested design 
\cite{robertson_conceptual_1971} of zone II-B has 8 irregularly-shaped 
graphite elements as well as dozens of salt channels. These graphite elements 
were simplified into right-circular cylindrical shapes with central channels. 
Figure~\ref{fig:serpent_zoneII} illustrates this core region in the Serpent 
model. The volume of fuel salt in zone II was kept exactly at 37\%, so that 
this simplification did not considerably change the core neutronics. 
Simplyfying the eight edge channels was the only simplification made to the 
\gls{MSBR} geometry in this work. 
\begin{figure}[ht!] % replace 't' with 'b' to \centering
	\includegraphics[width=\textwidth]{ch3/zone_II_element_ref.png}
	\caption{Graphite moderator elements for zone II-A: reference design (left)
		\cite{robertson_conceptual_1971} and Serpent model (right) 
		\cite{rykhlevskii_full-core_2017}.  Yellow 
		represents fuel salt, purple represents graphite, and aqua represents 
		the reactor vessel. Figure reproduced from Rykhlevskii \emph{et al.} 
		\cite{rykhlevskii_modeling_2019}.}
	\label{fig:II_element_ref}
\end{figure}

\subsection{Material composition and normalization parameters}
The fuel salt, reactor graphite, and modified Hastelloy-N
are all materials created at \gls{ORNL} specifically for the \gls{MSBR}.
The initial fuel salt used the same density (3.35 g/cm$^3$) and composition 
LiF-BeF$_2$-ThF$_4$-$^{233}$UF$_4$ (71.75-16-12-0.25 mole \%) as the 
\gls{MSBR} design \cite{robertson_conceptual_1971}. The lithium in the molten 
salt fuel is fully enriched to 100\% $^{7}$Li because $^{6}$Li is a very 
strong neutron poison and becomes tritium upon neutron capture. 

The specific temperature was fixed for each material and did not change during 
the reactor operation. The isotopic composition of each material at the 
initial state was described in detail in the MSBR conceptual design study 
\cite{robertson_conceptual_1971} and has been applied to the Serpent model 
without any modification. Table~\ref{tab:msbr_tab} is a summary of the major 
\gls{MSBR} parameters used to inform Serpent model  
\cite{robertson_conceptual_1971}. 
%%%%%%%%%%%%%%%%%%%%%%%%%%%%%%%%%%%%%%%%
\begin{table}[h!]
	\caption{Summary of principal data for \gls{MSBR} 
		\cite{robertson_conceptual_1971}.}
	\begin{tabularx}{\textwidth}{ X  X}
		\hline
		Thermal power           		& 2250 MW$_{th}$		\\
		Electric power             		& 1000 MW$_e$           \\
		Gross thermal efficiency       	& 44.4\%         		\\
		Salt volume fraction in central zone I		& 0.13   	\\
		Salt volume fraction in outer zone II       & 0.37		\\
		Fuel salt inventory (Zone I)                & 8.2 m$^3$	\\
		Fuel salt inventory (Zone II)               & 10.8 m$^3$\\
		Fuel salt inventory (annulus)               & 3.8 m$^3$	\\
		Total fuel salt inventory                   & 48.7 m$^3$\\
		Fissile mass in fuel salt                   & 1303.7 kg	\\
		Fuel salt components   	& LiF-BeF$_2$-ThF$_4$-$^{233}$UF$_4$	\\  
		Fuel salt composition   & 71.75-16-12-0.25 mole\%		\\
		Fuel salt density       & 3.35 g/cm$^3$         		\\ \hline
	\end{tabularx}
	\label{tab:msbr_tab}
\end{table}
%%%%%%%%%%%%%%%%%%%%%%%%%%%%%%%%%%%%%%%%%%%%%%%%

As mentioned in section~\ref{sec:reproc-plant}, the \gls{MSBR} design 
requires online reprocessing to remove neutron gaseous \glspl{FP} (Xe, Kr) and 
noble metals (e.g., Se, Nb, Mo) every 20 seconds.  The $^{232}$Th in the fuel 
absorbs thermal neutrons and produces $^{233}$Pa which then decays into the 
fissile $^{233}$U. Protactinium presents a challenge, since it has a large 
absorption cross section in the thermal energy spectrum. Moreover, $^{233}$Pa 
left in the core would produce $^{234}$Pa and $^{234}$U, neither of which are 
useful as fuel. Accordingly, $^{233}$Pa is continuously removed from the fuel 
salt into a protactinium decay tank to allow $^{233}$Pa to decay to $^{233}$U 
without the corresponding negative neutronic impact. The reactor chemical 
processing system must separate $^{233}$Pa from the molten salt fuel over 3 
days, hold it while $^{233}$Pa decays into $^{233}$U, and return it back to 
the primary loop. This feature allows the reactor to avoid neutron losses to 
protactinium, lowers in-core fission product inventory, and increases the 
efficiency of $^{233}$U breeding.

Table~\ref{tab:reprocessing_list_msbr} summarizes a full list of nuclides and 
their cycle time used for modeling salt treatment and separations 
\cite{robertson_conceptual_1971}. The removal rates vary among chemical 
elements in this reactor concept and dictate the necessary resolution of 
depletion calculations. If the depletion time intervals are very short, an 
enormous number of depletion steps are required to obtain the equilibrium 
composition. On the other hand, if the depletion  calculation time interval is 
too long, the impact of short-lived fission products is not captured. To 
compromise, a 3-day time interval was selected for depletion calculations to 
correlate with the removal interval of $^{233}$Pa, and $^{232}$Th was 
continuously added to maintain the initial mass fraction of $^{232}$Th.
%%%%%%%%%%%%%%%%%%%%%%%%%%%%%%%%%%%%%%%%
\begin{table}[ht!]
	\caption{The cycle times for protactinium and fission 
		products removal from the \gls{MSBR} (reproduced from Robertson 
		\emph{et al.} 
		\cite{robertson_conceptual_1971}).}
	\begin{tabularx}{\textwidth}{x  s  x}
		\hline \textbf{Processing group} & \qquad\qquad\qquad 
		\textbf{Nuclides} & \textbf{Cycle time (at full power)} \\ \hline 
		Rare earths & Y, La, Ce, Pr, Nd, Pm, Sm, 
		Gd & 50 days \\ \qquad & Eu & 500 days \\ Noble metals & Se, 
		Nb, Mo, Tc, Ru, Rh, Pd, Ag, Sb, Te & 20 sec \\
		Seminoble metals & Zr, Cd, In, Sn & 200 days \\
		Gases & Kr, Xe & 20 sec \\ Volatile fluorides & Br, I & 60 days \\
		Discard & Rb, Sr, Cs, Ba & 3435 days \\ 
		%Salt discard & Th, Li, Be, F & 3435 days \\ 
		Protactinium & $^{233}$Pa & 3 days \\ Higher 
		nuclides & $^{237}$Np, $^{242}$Pu & 16 years \\  \hline
	\end{tabularx}
	\label{tab:reprocessing_list_msbr}
\end{table}
%%%%%%%%%%%%%%%%%%%%%%%%%%%%%%%%%%%%%%%%%

\section{Fuel salt isotopic composition dynamics and equilibrium search}
The SaltProc online reprocessing simulation package is demonstrated in four 
applications: (1) analyzing the \gls{MSBR} neutronics and fuel cycle to find 
the equilibrium core composition and fuel salt depletion, (2) studying 
operational and safety parameters evolution during \gls{MSBR} operation, (3) 
demonstrating that in a single-fluid two-region \gls{MSBR} conceptual design 
the undermoderated outer core zone II works as a virtual ``blanket'', reduces 
neutron leakage and improves breeding ratio due to neutron energy spectral 
shift, and (4) determining the effect of fission product removal on the core 
neutronics.

The neutron population per cycle and the number of active/inactive cycles were 
chosen to obtain balance between reasonable uncertainty for a transport 
problem ($\leq$ 15 pcm\footnote{ $1$ $pcm$ = 10$^{-5}\Delta k_{eff}/k_{eff}$} 
for effective multiplication factor) and computational time. The \gls{MSBR} 
depletion and safety parameter computations were performed on 64 Blue Waters 
XK7 nodes (two AMD 6276 Interlagos CPU per node, 16 floating-point Bulldozer 
core units per node or 32 ``integer'' cores per node, nominal clock speed is 
2.45 GHz). The total computational time for calculating the equilibrium 
composition was approximately 9,900 node-hours (18 core-years.)

\subsection{Effective multiplication factor dynamics}
Figures~\ref{fig:keff}, \ref{fig:keff_zoomed} show the effective multiplication 
factors obtained using SaltProc and Serpent. The effective multiplication 
factors were calculated after removing fission products listed 
in  Table~\ref{tab:reprocessing_list_msbr} and adding the fertile material at 
the end of each depletion step (3 days). The effective multiplication factor 
fluctuates significantly as a result of the batch-wise nature of this online 
reprocessing strategy. 
\begin{figure}[ht!] 
	\centering
	\includegraphics[width=\textwidth]{ch3/keff.png}
	\caption{Effective multiplication factor dynamics for full-core \gls{MSBR} 
		model over a 60-year reactor operation lifetime. Figure reproduced 
		from Rykhlevskii \emph{et al.} \cite{rykhlevskii_modeling_2019}.}
	\label{fig:keff}
\end{figure}
\begin{figure}[ht!] 
	\centering
	\includegraphics[width=\textwidth]{ch3/keff_zoomed.png}
	\caption{Zoomed effective multiplication factor for 150-EFPD time 
	interval. Figure reproduced from Rykhlevskii \emph{et al.}  
	\cite{rykhlevskii_modeling_2019}.}
	\label{fig:keff_zoomed}
\end{figure}

First, Serpent calculates the effective multiplication factor for the  
beginning of the cycle (there is fresh fuel composition at the first step). 
Next, it computes the new fuel salt composition at the end of a 3-day 
depletion. The corresponding effective multiplication factor is much smaller 
than the previous one. Finally, SERPENT calculates $k_{eff}$ for the depleted 
composition after applying feeds and removals. The $k_{eff}$ increases 
accordingly since major reactor poisons (e.g. Xe, Kr) are removed, while fresh 
fissile material ($^{233}$U) from the protactinium decay tank is added.  

Additionally, the presence of rubidium, strontium, cesium, and barium in the 
core are disadvantageous to reactor physics. Overall, the effective 
multiplication factor gradually decreases from 1.075 to $\approx$1.02 at 
equilibrium after approximately 6 years of irradiation. 

%In fact, SaltProc v0.1 fully removes all of these elements every 3435 days 
%(not a small mass fraction every 3 days) which causes the multiplication 
%factor to jump by approximately $450$ $pcm$, and limits using the batch 
%approach for online reprocessing simulations. In SaltProc v1.0 this drawback 
%has been eliminated by removing fraction of element at each depletion step 
%with 
%longer residence times (seminoble metals, volatile fluorides, Rb, Sr, Cs, Ba, 
%Eu). Results obtain with this approach will be presented for the \gls{TAP} 
%\gls{MSR} in chapter 4.


\subsection{Fuel salt composition dynamics}
The analysis of the fuel salt composition evolution provides more comprehensive 
information about the equilibrium state. Figure~\ref{fig:adens_eq} shows the 
number densities of major nuclides which have a strong influence on the reactor 
core physics. The concentration of $^{233}$U, $^{232}$Th, $^{233}$Pa, and 
$^{232}$Pa in the fuel salt change insignificantly after approximately 2500 
days of operation. In particular, the $^{233}$U number density fluctuates by 
less than 0.8\% between 16 and 20 years of operation. Hence, a 
quasi-equilibrium state was achieved after 16 years of reactor operation.
\begin{figure}[ht!] % replace 't' with 'b' to 
	\centering
	\includegraphics[width=\textwidth]{ch3/major_isotopes_adens.png}
	\caption{Number density of major nuclides during 60 years of reactor 
		operation.}
	\label{fig:adens_eq}
\end{figure}
In contrast, a wide variety of nuclides, including fissile isotopes (e.g. 
$^{235}$U) and non-fissile strong absorbers (e.g. $^{234}$U), kept accumulating 
in the core. Figure~\ref{fig:fissile_short} demonstrates production of fissile 
isotopes in the core. In the end of the considered operational time, the core 
contained significant $^{235}$U ($\approx10^{-5}$ atom/b-cm), $^{239}$Pu 
($\approx5\times10^{-7}$ atom/b-cm), and $^{241}$Pu ($\approx 5\times10^{-7}$ 
atom/b-cm). Meanwhile, the equilibrium number density of the target fissile 
isotope $^{233}$U was approximately 7.97$\times10^{-5}$ atom/b-cm. Small dips 
in neptunium and plutonium number density every 16 years are caused by removing
$^{237}$Np and $^{242}$Pu (included in Processing group ``Higher nuclides'', see
Table~\ref{tab:reprocessing_list}) which decay into $^{235}$Np and $^{239}$Pu, 
respectively. Thus, production of new fissile materials in the core, as well as 
$^{233}$U breeding, made it possible to compensate for negative effects of 
strong absorber accumulation and keep the reactor critical.
\begin{figure}[htp!] % replace 't' with 'b' to 
	\centering
	\includegraphics[width=\textwidth]{ch3/fissile_short.png}
	\caption{Number density of fissile in epithermal spectrum nuclides 
		accumulation during the reactor operation.}
	\label{fig:fissile_short}
\end{figure}

\subsection{Neutron spectrum}
Figure~\ref{fig:spectrum} shows the normalized neutron flux spectrum for the 
full-core \gls{MSBR} model in the energy range from $10^{-8}$ to $10$ MeV. The 
neutron energy spectrum at equilibrium is harder than at startup due to 
plutonium and other strong absorbers accumulating in the core during reactor 
operation.  
\begin{figure}[ht!] % replace 't' with 'b'         to force it to 
	\centering
	\includegraphics[width=\textwidth]{ch3/spectrum.png}
	\caption{The neutron flux energy spectrum is normalized by unit lethargy 
	and the area under the curve is normalized to 1 for initial and equilibrium 
	fuel salt composition.}
	\label{fig:spectrum}
\end{figure}

Figure~\ref{fig:spectrum_zones} shows that zone I produced more thermal  
neutrons than zone II, corresponding to a majority of fissions occurring in the 
central part of the core. In the undermoderated zone II, the neutron energy 
spectrum is harder, which leads to more neutrons capture by $^{232}$Th and 
helps achieve relatively high breeding ratio. Moreover, the (n,$\gamma$) 
resonance energy range in $^{232}$Th is from 10$^{-4}$ to 10$^{-2}$ MeV. 
Therefore, the moderator-to-fuel ratio for zone II was chosen to shift the 
neutron energy spectrum in this range. Furthermore, in the central core region 
(zone I), the neutron energy spectrum shifts to a harder spectrum over 20 years 
of reactor operation. Meanwhile, in the outer core region (zone II), a 
similar spectral shift takes place at a reduced scale. These results are in a 
good agreement with original ORNL report \cite{robertson_conceptual_1971} and 
the most recent whole-core steady-state study \cite{park_whole_2015}.

It is important to obtain the epithermal and thermal spectra to produce 
$^{233}$U from $^{232}$Th because the radiative capture cross section of 
thorium decreases monotonically from $10^{-10}$ MeV to $10^{-5}$ MeV. Hardening 
the spectrum tends to significantly increase resonance absorption in thorium 
and decrease absorptions in fissile and construction materials. 
\begin{figure}[ht!] % replace 't' with 'b' to force it to 
	\centering
	\includegraphics[width=\textwidth]{ch3/spectrum_zones.png} 
	\caption{The neutron flux energy spectrum in different core regions is 
	normalized by unit lethargy and the area under the curve is normalized to 1 
	for the initial and equilibrium fuel salt composition.}
	\label{fig:spectrum_zones}
\end{figure}

\subsection{Neutron flux}
Figure~\ref{fig:radial_flux} shows the radial distribution of fast and thermal 
neutron flux for the both initial and equilibrium composition. The neutron 
fluxes have similar shapes for both compositions but the equilibrium case has a 
harder spectrum. A significant spectral shift was observed in the central 
region of the core (zone I), while for the outer region (zone II), it is 
negligible for fast but notable for thermal neutrons. These neutron flux radial 
distributions agree with the fluxes in the original ORNL report 
\cite{robertson_conceptual_1971}. 
Overall, spectrum hardening during \gls{MSBR} operation should be carefully 
studied when designing the reactivity control system.
\begin{figure}[ht!] % replace 't' with 'b' to force it to \centering
	\includegraphics[width=\textwidth]{ch3/radial_flux.png}
	\caption{Radial neutron flux distribution for initial and equilibrium fuel 
	salt composition.}
	\label{fig:radial_flux}
\end{figure}

\subsection{Power and breeding distribution}
Table~\ref{tab:powgen_fraction} shows the power fraction in each zone for 
initial and equilibrium fuel compositions. Figure~\ref{fig:pow_den} reflects 
the normalized power distribution of the \gls{MSBR} quarter core for 
equilibrium fuel salt composition. For both the initial and equilibrium 
compositions, fission primarily occurs in the center of the core, namely zone 
I. The spectral shift during reactor operation results in slightly different 
power fractions at startup and equilibrium, but most of the power is still 
generated in zone I at equilibrium (table~\ref{tab:powgen_fraction}). 
%%%%%%%%%%%%%%%%%%%%%%%%%%%%%%%%%%%%%%%%
\begin{table}[ht!]
	\caption{Power generation fraction in each zone for initial and equilibrium 
		state.}
	\begin{tabularx}{\textwidth}{ m | s | s } \hline
		Core region      & Initial      & Equilibrium   \\   \hline
		Zone I           & 97.91\%      & 98.12\%   \\
		Zone II          & 2.09\%       & 1.88\%   \\ \hline
	\end{tabularx}
	\label{tab:powgen_fraction}
\end{table}
%%%%%%%%%%%%%%%%%%%%%%%%%%%%%%%%%%%%%%%%%%%%%%%%%%%%%%%%%%%%%%%%%%%%%%%%%%%%%%%%
Figure~\ref{fig:breeding_den} shows the neutron capture reaction rate 
distribution for $^{232}$Th normalized by the total neutron flux for initial 
and equilibrium states. The distribution reflects the spatial distribution of 
$^{233}$U production in the core. $^{232}$Th neutron capture produces 
$^{233}$Th which then $\beta$-decays to $^{233}$Pa, the precursor for $^{233}$U 
production. Accordingly, this characteristic represents the breeding 
distribution in the \gls{MSBR} core. Spectral shift does not cause significant 
changes in power nor in breeding distribution. Even after 20 years of 
operation, most of the power is still generated in zone I.
\begin{figure}[ht!] % replace 't' with 'b' to force it to 
	\centering
	\includegraphics[width=0.82\textwidth]{ch3/power_distribution_eq.png} 
		\vspace{-4mm}
	\caption{Normalized power density for equilibrium fuel salt 
		composition.}
	\label{fig:pow_den}
\end{figure}
	\vspace{-2mm}
\begin{figure}[ht!] % replace 't' with 'b' to force it to 
	\centering
	\includegraphics[width=0.82\textwidth]{ch3/breeding_distribution_eq.png} 
		\vspace{-4mm}
	\caption{$^{232}$Th neutron capture reaction rate normalized by total flux 
		for equilibrium fuel salt composition.}
	\label{fig:breeding_den}
\end{figure}
\FloatBarrier

%%%%%%%%%%%%%%%%%%%%%%%%%%%%%%%%%%%%%%%%%%%%%%%%%%%%%%%%%%%%%%%%%%%%%%%%%%%%%%%%
\subsection{Thorium refill rate}
In a \gls{MSBR} reprocessing scheme, the only external feed material flow  is 
$^{232}$Th. Figure~\ref{fig:th_refill} shows the $^{232}$Th feed rate 
calculated for 60 years of reactor operation. The $^{232}$Th feed rate 
fluctuates significantly as a result of the batch-wise nature of this online 
reprocessing approach. Figure~\ref{fig:th_refill_zoomed} shows zoomed thorium 
feed rate for short 150-EFPD interval. Note that the large spikes of up to 36 
kg/day in a thorium consumption occurs every 3435 days. This is required due to 
strong absorbers (Rb, Sr, Cs, Ba) removal at the end of effective cycle (100\% 
of these elements removing every 3435 days of operation). The corresponding 
effective multiplication factor increase (Figure~\ref{fig:keff}) and breeding 
intensification leads to additional $^{232}$Th consumption.  
\begin{figure}[ht!] % replace 't' with 'b' to force it to \centering
	\includegraphics[width=\textwidth]{ch3/th_refill_rate.png} 
	\caption{$^{232}$Th feed rate over 60 years of \gls{MSBR} operation.}
	\label{fig:th_refill}
\end{figure}
\begin{figure}[ht!] % replace 't' with 'b' to force it to \centering
	\includegraphics[width=\textwidth]{ch3/th_refill_rate_zoomed.png} 
	\caption{Zoomed $^{232}$Th feed rate for 150-EFPD time interval.}
	\label{fig:th_refill_zoomed}
\end{figure}

The average thorium feed rate increases during the first 500 days of operation, 
and steadily decreases due to spectrum hardening and accumulation of absorbers 
in the core. As a result, the average $^{232}$Th feed rate over 60 years of 
operation is about 2.40 kg/day. This thorium consumption rate is in good 
agreement with a recent online reprocessing study by \gls{ORNL} 
\cite{betzler_molten_2017}. At equilibrium, the thorium feed rate is determined 
by the reactor power, the energy released per fission, and the neutron energy 
spectrum.


\section{Operational and safety parameters evolution}
\subsection{Temperature coefficient of reactivity}
Table~\ref{tab:tcoef} summarizes temperature effects on reactivity calculated 
in this work for both initial and equilibrium fuel compositions, compared 
with the original \gls{ORNL} report data \cite{robertson_conceptual_1971}. 
By propagating the $k_{eff}$  statistical error provided by SERPENT2, 
uncertainty for each temperature coefficient was obtained and appears in 
Table~\ref{tab:tcoef}. Other sources of uncertainty are neglected, such as 
cross section measurement error and approximations inherent in the equations of 
state providing both the salt and graphite density dependence on temperature.
The main physical principle underlying the reactor temperature feedback is an 
expansion of heated material. When the fuel salt temperature increases, the 
density of the salt decreases, but at the same time, the total volume of fuel 
salt in the core remains constant because it is bounded by the graphite. When 
the graphite temperature increases, the density of graphite decreases, creating 
additional space for fuel salt. To determine the temperature coefficients, the 
cross section temperatures for the fuel and moderator were changed from 900K to 
1000K. Three different cases were considered:
\begin{enumerate}
	\item Temperature of fuel salt rising from 900K to 1000K.
	\item Temperature of graphite rising from 900K to 1000K.
	\item Whole reactor temperature rising from 900K to 1000K.
\end{enumerate}
%%%%%%%%%%%%%%%%%%%%%%%%%%%%%%%%%%%%%%%%
\begin{table}[ht!]
	\caption{Temperature coefficients of reactivity for initial and equilibrium 
		state.}
	\begin{tabularx}{\textwidth}{ X | r | r | r } \hline
		Reactivity coefficient               & Initial         & 
		Equilibrium     & Reference                                 \\ 
		& [pcm/k]         &  [pcm/k]        & 
		(initial)\cite{robertson_conceptual_1971} \tabularnewline  \hline
		Doppler in fuel salt                    & $-4.73\pm0.038$ & 
		$-4.69\pm0.038$ & $-4.37$  \tabularnewline
		Fuel salt density                       & $+1.21\pm0.038$ & 
		$+1.66\pm0.038$ & $+1.09$  \tabularnewline
		Total fuel salt                         & $-3.42\pm0.038$ & 
		$-2.91\pm0.038$ & $-3.22$  \tabularnewline \hline
		Graphite spectral shift                 & $+1.56\pm0.038$ & 
		$+1.27\pm0.038$ &          \tabularnewline
		Graphite density                        & $+0.14\pm0.038$ & 
		$+0.23\pm0.038$ &          \tabularnewline
		Total moderator (graphite)              & $+1.69\pm0.038$ & 
		$+1.35\pm0.038$ & $+2.35$  \tabularnewline \hline
		Total core                              & $-1.64\pm0.038$ & 
		$-1.58\pm0.038$ & $-0.87$  \tabularnewline \hline
	\end{tabularx}
	\label{tab:tcoef}
\end{table}
%%%%%%%%%%%%%%%%%%%%%%%%%%%%%%%%%%%%%%%%%%%%%%%%%%%%%%%%%%%%%%%%%%%%%%%%%%%%%%%%
In the first case, changes in the fuel temperature only impact fuel density. In 
this case, the geometry is unchanged because the fuel is a liquid. However, 
when the moderator heats up, both the density and the geometry change due to 
thermal expansion of the solid graphite blocks and reflector. Accordingly, the 
new graphite density was calculated using a linear temperature expansion 
coefficient of 1.3$\times10^{-6}$K$^{-1}$ \cite{robertson_conceptual_1971}. A 
new geometry input for SERPENT2, which takes into account displacement of 
graphite surfaces, was created based on this information. For calculation of 
displacement, it was assumed that the interface between the graphite reflector 
and vessel did not move, and that the vessel temperature did not change. This 
is the most reasonable assumption for the short-term reactivity effects because 
inlet salt is cooling graphite reflector and inner surface of the vessel.

The fuel temperature coefficient (FTC) is negative for both initial and 
equilibrium fuel compositions due to thermal Doppler broadening of the 
resonance capture cross sections in the thorium. A small positive effect of 
fuel density on reactivity increases from $+1.21$ pcm/K at reactor startup to 
$+1.66$ pcm/K for equilibrium fuel composition which has a negative effect on 
FTC magnitude during the reactor operation. This is in good agreement with 
earlier research \cite{robertson_conceptual_1971,park_whole_2015}. The 
moderator temperature coefficient (MTC) is positive for the startup composition 
and decreases during reactor operation because of spectrum hardening with fuel 
depletion. Finally, the total temperature coefficient of reactivity is negative 
for both cases, but decreases during reactor operation due to spectral shift. 
In summary, even after 20 years of operation the total temperature coefficient 
of reactivity is relatively large and negative during reactor operation 
(comparing with conventional PWR which has temperature coefficient about -1.71 
pcm/$^\circ$F $\approx$ -3.08 pcm/K \cite{forget_integral_2018}), despite 
positive MTC, and affords excellent reactor stability and control.

\subsection{Reactivity control system rod worth}
Table~\ref{tab:rod_worth} summarizes the reactivity control system worth. 
During normal operation, the control (graphite) rods are fully inserted, and 
the safety (B$_4$C) rods are fully withdrawn. To insert negative reactivity 
into the core, the graphite rods are gradually withdrawn from the core. In an 
accident, the safety rods would be dropped down into the core. The integral rod 
worths were calculated for various positions to separately estimate the worth
of the control graphite rods\footnote{In \cite{robertson_conceptual_1971}, the 
graphite rods are referred to as ``control'' rods.}, the safety (B$_4$C) rods, 
and the whole reactivity control system. Control rod integral worth is 
approximately 28 cents and stays almost constant during reactor operation. The 
safety rod integral worth decreases by  16.2\% during 20 years of operation 
because of neutron spectrum hardening and absorber accumulation in proximity to 
reactivity control system rods. This 16\% decline in control system worth 
should be taken into account in \gls{MSBR} accident analysis and safety 
justification.
%%%%%%%%%%%%%%%%%%%%%%%%%%%%%%%%%%%%%%%%
\begin{table}[ht!]
	\caption{Control system rod worth for initial and equilibrium fuel 
		composition.}
	\begin{tabularx}{\textwidth}{ b | x | x } \hline
		Reactivity parameter [cents]  &  Initial      &  Equilibrium      \\ 
		\hline
		Control (graphite) rod integral worth               & $\ 28.2\pm0.8$    
		& $\ 
		29.0\pm0.8$ \\ Safety (B$_4$C) rod integral worth                  & 
		$251.8\pm0.8$    & $211.0\pm0.8$  \\
		Total reactivity control system worth               & $505.8\pm0.7$    
		& 
		$424.9\pm0.8$ \\ \hline
	\end{tabularx}
	\label{tab:rod_worth}
\end{table}
%%%%%%%%%%%%%%%%%%%%%%%%%%%%%%%%%%%%%%%%%%%%%%%%%%%%%%%%%%%%%%%%%%%%%%%%%%%%%%%%

\subsection{Six Factor Analysis}
The effective multiplication factor can be expressed using the following 
formula:
\begin{align*}
k_{eff} = k_{inf} P_f  P_t = \eta \epsilon p f P_f P_t
\end{align*}

Table~\ref{tab:six_factor} summarizes the six factors for both initial and 
equilibrium fuel salt composition. Using SERPENT2 and SaltProc, these factors 
and their statistical uncertainties have been calculated for both initial and 
equilibrium fuel salt composition (see Table~\ref{tab:msbr_tab}). The fast and 
thermal non-leakage probabilities remain constant despite the evolving neutron 
spectrum during operation. In contrast, the neutron reproduction factor 
($\eta$), resonance escape probability ($p$), and fast fission factor 
($\epsilon$) are considerably different between startup and equilibrium. As 
indicated in Figure~\ref{fig:spectrum}, the neutron spectrum is softer at the 
beginning of reactor life. Neutron spectrum hardening causes the fast fission 
factor to increase through the core lifetime. The opposite is true for the 
resonance escape probability. Finally, the neutron reproduction factor 
decreases during reactor operation due to accumulation of fissile plutonium 
isotopes.
%%%%%%%%%%%%%%%%%%%%%%%%%%%%%%%%%%%%%%%%
\begin{table}[hb!]
	\caption{Six factors for the full-core \gls{MSBR} model for initial and 
		equilibrium fuel composition.}
	\begin{tabularx}{\textwidth}{ b | s | s } \hline
		Factor  & Initial      & Equilibrium   \\ \hline
		Neutron reproduction factor ($\eta$)     & $1.3960\pm.000052$     & 
		$1.3778\pm.00005$ \\ Thermal utilization factor (f)           & 
		$0.9670\pm.000011$     & $0.9706\pm.00001$ \\
		Resonance escape probability (p)         & $0.6044\pm.000039$     & 
		$0.5761\pm.00004$ \\
		Fast fission factor ($\epsilon$)         & $1.3421\pm.000040$     & 
		$1.3609\pm.00004$ \\
		Fast non-leakage probability (P$_f$)     & $0.9999\pm.000004$     & 
		$0.9999\pm.000004$ \\
		Thermal non-leakage probability (P$_t$)  & $0.9894\pm.000005$     & 
		$0.9912\pm.00005$ \\ \hline
	\end{tabularx}
	\label{tab:six_factor}
\end{table}

\section{Benefits of fission product removal}
\subsection{The effect of removing fission product from fuel salt}
Loading initial fuel salt composition into the \gls{MSBR} core leads to a 
supercritical configuration (Figure ~\ref{fig:fp_removal}). After reactor 
startup, the effective multiplication factor for the case with volatile gases 
and noble metals removal is approximately 7500 pcm  higher than for case with 
no fission products removal. This significant impact on the reactor core is
achieved due to immediate removal (20 sec cycle time) and high absorption cross 
section of Xe, Kr, Mo, and other noble metals removed. The effect of rare earth 
element removal is considerable a few months after startup and reached 
approximately 5500 pcm after 10 years of operation. The rare earth elements 
were removed at a slower rate (50-day cycle time). Moreover,  
Figure~\ref{fig:fp_removal} demonstrates that batch-wise removal of strong 
absorbers every 3 days did not necessarily leads to fluctuation in results 
but rare earth elements removal every 50 days causes an approximately 600 pcm 
jump in reactivity.

The effective multiplication factor of the core reduces gradually over 
operation time because the fissile material ($^{233}$U) continuously depletes 
from the fuel salt due to fission while fission products accumulate in the fuel 
salt simultaneously. Eventually, without fission products removal, the 
reactivity decreases to the subcritical state after approximately 500 and 
1300 days of operation for cases with no removal and volatile gases \& noble 
metals removal, respectively. The time when the simulated core reaches 
subcriticality ($k_{eff}<$1.0) for full-core model) is called the core 
lifetime. Therefore, removing fission products provides with significant 
neutronic benefit and enables a longer core lifetime.
\begin{figure}[ht!] % replace 't' with 'b' to force it to 
	\centering
	\includegraphics[width=\textwidth]{ch3/keff_rem_cases.png} 
	\caption{Calculated effective multiplication factor for full-core 
	\gls{MSBR} model with removal of various fission product groups over 10 
	years of operation.}
	\label{fig:fp_removal}
\end{figure}


\section{Concluding remarks}