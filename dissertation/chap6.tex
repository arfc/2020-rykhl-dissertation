\chapter{Tool demonstration for load-following and safety analysis: Molten 
Salt Breeder Reactor}
Previous chapter shown that the \gls{TAP} \gls{MSR} is unaffected by the xenon 
poisoning effect during power variation because it has relatively fast neutron 
energy spectrum. 
While long-term performance 
metrics such as fuel utilization would definitely benefit from online removal 
of poisonous fission products, the gas removal system is unessential to ensure 
the \gls{TAP} system operation during short-term transient with significant 
power drop and following restart. However, Chapter 5 clearly demonstrated the 
noble gas removal strong impact on the reactor neutronics during power 
adjustments. Thus, another liquid-fueled \gls{MSR} design with thermal 
spectrum (not epithermal like in the \gls{TAP} core) should be considered to 
investigate benefits of the online gas removal for load-following 
operation.	

This chapter presents fuel salt depletion analysis with SaltProc during a
short-term power transient to evaluate load-following capabilities of the 
graphite-moderated molten salt reactor design with a thermal neutron energy 
spectrum - \gls{MSBR}. The details of the \gls{MSBR} design, the full-core 
Serpent model, and 
results of long-term depletion simulation with SaltProc are described in 
Chapter 3. I simulate the load-following transient postulated in 
Section~\ref{sec:worst-load} using methodology described in Chapter 5. To 
investigate the effect of noble gas 
removal efficiency on load-following operation, I consider three various 
regimes of the gas removal system operation:
\begin{enumerate}[label=(\alph*), noitemsep, topsep=0pt]
	\item no gas removal ($\epsilon_{Xe}=0.0$);
	\item moderate gas removal efficiency ($\epsilon_{Xe}=0.536$);
	\item high gas removal efficiency ($\epsilon_{Xe}=0.915$).
\end{enumerate}
I then calculate a major safety and operational parameters for all three 
regimes at various moments of the transient to make sure that the critical 
safety margins are maintained. Finally, I compare the \gls{TAP} \gls{MSR} and 
\gls{MSBR} behavior during the postulated load-following transient.


\section{Depletion analysis results}
Using the methodology described previously in Chapter 5, the \gls{MSBR} 
full-core depletion analysis was performed using SaltProc v1.0. I 
used 30-minutes depletion time step to capture rapid changes in reactivity. 
The Equation~\ref{eq:time-xe-max} predicted the time after shutdown when 
$^{135}$Xe concentration peaks ($t^{max}_X$) in the range from 6.8h 
($\epsilon_{Xe}=0.0$, 30 years after startup) to 7.5h ($\epsilon_{Xe}=0.915$, 
\gls{BOL}). The $t^{max}_X$ for the \gls{MSBR} is longer than 
for the \gls{TAP} reactor (2.75h) due to much more thermal neutron energy 
spectrum.
To be consistent throughout different gas removal regimes while 
investigating load-following capabilities of the \gls{MSBR}, I selected 
following transient (power load profile) very similar to the transient chosen 
in Chapter 5:
\begin{enumerate}[label=(\alph*), noitemsep, topsep=0pt]
	\item operate on 100\% of \gls{HFP} to reach $^{135}$I/$^{135}$Xe 
	equilibrium (at 
	least 3 days from the startup);
	\item instantaneous power drop from 100\% to 0\%;
	\item shutdown for $t^{max}_X=7.5h$ to reach the $^{135}$Xe concentration 
	extremum;
	\item restart the reactor instantly from 0\% to 100\% power level and 
	operate on 100\% for a 5 hours.
\end{enumerate}


\subsection{Reactivity dynamics}
Figures~\ref{fig:msbr-lf-keff-evo} and \ref{fig:msbr-lf-rho-evo} show the 
effective multiplication factor and reactivity dynamics for the 
different gas removal efficiency in the \gls{MSBR} during the transient, 
described earlier. For the no-removal case (Figure~\ref{fig:msbr-lf-keff-evo}, 
upper panel), the effective multiplication dropped after $t^{max}_X=7.5h$ by 
1457 $pcm$ and 1035 $pcm$ at \gls{BOL} and after 15 years of full-power 
operation, respectively. Thus, the Equation~\ref{eq:time-xe-max} correctly 
predicted the moment when the xenon poisoning effect maximized for no-removal 
case ($\epsilon_{Xe}=0$).
After the power ramp-up from 0\% to 100\%, the effective multiplication factor 
restored to its initial value in about 3 hours. Notably, maximum 
negative reactivity introduction due to $^{135}$Xe buildup after the 
\gls{MSBR} shutdown is very similar to the \gls{PWR} (both at startup): 1457 
$pcm$ and 1500 $pcm$ \cite{rykhlevskii_impact_2019}, respectively. 
Additionally, the xenon poisoning effect diminished toward the \gls{EOL} 
because the $^{135}$Xe concentration peak is larger for softer thermal 
spectrum (the \gls{MSBR} spectrum becomes \emph{harder} during  operation due 
to plutonium and other strong neutron absorbers accumulation in the fuel salt).
Finally, the effect of $^{135}$Xe poisoning is almost the same after 15 and 30 
years of operation because the fuel salt composition reached its equilibrium 
after about 16 years of full-power operation (see 
Section~\ref{sec:ch3-msbr-fuel-comp}).

Middle and lower plots in Figure~\ref{fig:msbr-lf-rho-evo} show reactivity 
change during the \gls{MSBR} shutdown for 7.5 hours and following power ramp 
up to 100\% for moderate ($\epsilon_{Xe}=0.536$) and high 
($\epsilon_{Xe}=0.915$) removal efficiency, respectively. In contrast with 
no gas removal, reactivity dropped during the 30-minutes interval after 
shutdown by 161 and 189 $pcm$ for moderate and high removal efficiency, 
respectively.  Afterward, the reactivity boosts by 1494 $pcm$ and 2608 $pcm$ 
for $\epsilon_{Xe}=0.536$ and 0.915, respectively. This happens because 
the gas removal system extracted 53.6\% and 91.5\% of xenon mass at the end of 
30-minutes depletion step. The more effective xenon removal leads to greater 
positive reactivity jump, as expected. Notably, the reactivity stabilizes at 
approximately $+2500$ $pcm$ level for both cases in about 5 hours from the 
beginning of the transient because $^{135}$Xe loss due to its decay and online 
gas removal equalizes $^{135}$Xe gain due to $^{135}$I decay.
Overall, the online gas removal from the fuel salt even with moderate 
efficiency is beneficial to the core neutronics and significantly reduces the 
xenon poisoning effect ($-161\pm10$ $pcm$ instead of $-1494\pm10$ $pcm$). 
Finally, the very high removal efficiency ($\epsilon_{Xe}=0.915$) is 
unnecessary to significantly reduce the effect of xenon poisoning and enable 
load-following capability of the \gls{MSBR}.
\begin{figure}[htbp!] % replace 't' with 'b' to 
	\centering
$\begin{array}{c}
	\includegraphics[width=0.92\textwidth]{ch6/kl1_keff.png}\vspace{-14mm}\\
	\includegraphics[width=0.905\textwidth]{ch6/kl25_keff.png}\vspace{-12mm}\\
	\includegraphics[width=0.905\textwidth]{ch6/kl100_keff.png}
\end{array}$
		\vspace{-5mm}
	\caption{SaltProc-calculated evolution of the effective multiplication 
	factor during the postulated load-following transient for various regimes 
	of the gas removal system operation. The uncertainty $\pm\sigma=10$ $pcm$ 
	is shaded.}
	\label{fig:msbr-lf-keff-evo}
\end{figure}

\begin{figure}[htbp!] % replace 't' with 'b' to 
	\centering
	$\begin{array}{r}
	\includegraphics[width=0.923\textwidth]{ch6/kl1_rho.png}\vspace{-14mm}\\
	\includegraphics[width=0.9\textwidth]{ch6/kl25_rho.png}\vspace{-13mm}\\
	\includegraphics[width=0.9\textwidth]{ch6/kl100_rho.png}
	\end{array}$
	\vspace{-5mm}
	\caption{SaltProc-calculated evolution of the reactivity during the 
	postulated load-following transient for various regimes 
		of the gas removal system operation. The uncertainty $\pm\sigma=10$ 
		$pcm$ 
		is shaded.}
	\label{fig:msbr-lf-rho-evo}
\end{figure}
\FloatBarrier

\subsection{Fuel salt composition evolution}
Figure~\ref{fig:msbr-lf-xe-i-ratio} shows $^{135}$Xe and $^{135}$I mass 
dynamics evolution during the postulated transient for various gas removal 
efficiencies. The $^{135}$I/$^{135}$Xe concentration ratio at the beginning 
transient for the no-removal case is 2.45 and 2.03 at the \gls{BOL} and after 
30 years of full-power operation, respectively. The greater 
$^{135}$I/$^{135}$Xe concentration ratio at the startup leads to $^{135}$Xe 
concentration peak by 11\% higher than at the \gls{EOL} which is consistent 
with the \gls{TAP} \gls{MSR} results. However, larger $^{135}$Xe concentration 
does not necessary worsens the xenon poisoning effect 
(Figure~\ref{fig:msbr-lf-rho-evo}) because the spectrum hardens toward 
\gls{EOL} and $^{135}$Xe absorption cross section slumps with energy (see
Figure~\ref{fig:tap-pwr-spectrum}).

For the high gas removal efficiency regime, the $^{135}$I/$^{135}$Xe 
concentration ratio is 2.47 and 2.08 at the \gls{BOL} and after 
30 years of full-power operation, respectively. For the \gls{BOL} and 
\gls{EOL}, the $^{135}$Xe concentration peaked only by 8\% at the end of a 
first 30-minute depletion step which caused a 189-$pcm$ negative reactivity 
insertion. Afterward, the concentration of $^{135}$Xe dropped quickly because 
the gas removal system extracted most of the fission gas. The $^{135}$Xe 
concentration in the fuel salt before the shutdown is approximately 7 times 
greater than after the power turned back on, which caused significant 
reactivity growth by $\approx2550$ $pcm$. Surprisingly, the removal of 12 g of 
$^{135}$Xe from $t=30min$ to $t=60min$ caused an impressive 2600-$pcm$ 
positive reactivity insertion (217 pcm/g$^{135}$Xe reactivity worth). 
\begin{figure}[htbp!] % replace 't' with 'b' to 
	\centering
	$\begin{array}{r}
	\includegraphics[width=0.88\textwidth]{ch6/kl1_xe_i_ratio.png}\vspace{-13mm}\\
	\includegraphics[width=0.88\textwidth]{ch6/kl25_xe_i_ratio.png}\vspace{-13mm}\\
	\includegraphics[width=0.88\textwidth]{ch6/kl100_xe_i_ratio.png}
	\end{array}$
	\vspace{-4mm}
	\caption{Comparison of $^{135}$Xe and $^{135}$I isotopic content at the 
		\gls{BOL} (dashed line) and after 30 years of operation (solid line) 
		for various gas removal regimes. Uncertainty of the predicted mass 
		will be estimated and discussed in Chapter 7.}
	\label{fig:msbr-lf-xe-i-ratio}
\end{figure}

Such a quick fluctuations in the $^{135}$Xe concentration are observed due to 
batch-wise nature 
of SaltProc simulations (e.g., the fraction of target poison is being 
removed discretely, at the end of each depletion step). Realistically, the gas 
removal system extracts gas from the fuel salt continuously, which would 
result in a much smoother change in the concentration and, accordingly, 
in the reactivity. Notably, for the both \gls{BOL} and \gls{EOL}, the 
$^{135}$Xe mass stabilized at 1 g in about 3-4 hours after the shutdown and 
then inclined slowly ($60$ $mg/EFPH$) after power ramp-up from 0 to 100\%. 
That is, when the reactor returned back to a full-power level,  the $^{135}$Xe 
concentration during a few days will be significantly lower than before 
load-following transient. Thus, less thermal neutrons will be parasitically 
absorbed in the fission gas. As a result, the long-term fuel cycle performance 
metrics such as fuel utilization and the core lifetime would benefit 
enormously from a ``clean-up" effect of the postulated transient, but such 
analysis is out of scope of this work.

In the case of moderate gas removal efficiency, major fission product 
concentration changes very similar to high removal efficiency case. The 
$^{135}$I/$^{135}$Xe concentration ratio is 2.15 and 2.06 at the \gls{BOL} and 
after 30 years of full-power operation, respectively, and caused 7.5\% hike 
in $^{135}$Xe concentration. Surprisingly, significantly lower gas removal 
efficiency ($\epsilon_{Xe}=0.536$ instead of 0.915) provided comparable 
benefits to the core neutronics during the postulated load-following 
transient. In fact, similarly to the $\epsilon_{Xe}=0.915$ case, the 
$^{135}$Xe mass stabilized at 1.5 g in about 5 hours after the shutdown and 
then increased slowly ($165$ $mg/EFPH$) after power ramp-up from 0 to 100\%.
In conclusion, simpler and cheaper gas removal system with extraction 
efficiency $\epsilon_{Xe}=0.536$ is good enough to suppress the xenon 
poisoning effect to acceptable level (-161 $pcm$) and ensure load-following 
capability of the \gls{MSBR}.


\subsection{Neutron spectrum}
Figure~\ref{fig:ch6-msbr-spectrum} shows that the \gls{MSBR} spectrum after 30 
years of operation (solid line) is harder than at the startup (dashed line). 
Compared with the \gls{MSBR}, the \gls{TAP} \gls{MSR} spectrum is 
significantly 
harder even when all moderator rods are inserted to the core. Notably, the 
\gls{MSBR} spectrum has clear peak in thermal energy region, but flat neutron 
energy dependence in intermediate and fast energy region, which is quite 
common for thermal reactors. In contrast, the \gls{TAP} core spectrum at the 
\gls{EOL} has high peak in fast and lower peak in thermal energy region, 
which is typical for epithermal/intermediate reactors. This is main reason, 
why for the postulated load-following transient I observed significant xenon 
poisoning effect in the \gls{MSBR} and negligible xenon impact in the 
\gls{TAP} \gls{MSR} (see Chapter 5).
\begin{figure}[htbp!] % replace 't' with 'b' to 
	\centering
	\includegraphics[width=\textwidth]{ch6/msbr_vs_tap_spectrum.png}
	\caption{Neutron spectra normalized by lethargy for the \gls{MSBR} and 
		\gls{TAP} at various moments during operation. The neutron flux 
		uncertainties $\sigma_{\Phi}$ are 0.6\% and 0.18\% for the \gls{TAP} 
		reactor and \gls{MSBR}, respectively.}
	\label{fig:ch6-msbr-spectrum}
\end{figure}

Potentially, any graphite-moderated liquid-fueled \gls{MSR} conceptual 
design\footnote{Integral Molten Salt Reactor (IMSR) from Terrestial Energy 
\cite{leblanc_integral_nodate}, Molten Salt Demonstration Reactor (MSDR) from 
Oak Ridge National Laboratory \cite{bettis_design_1972}, Liquid fluoride 
thorium reactor (LFTR) from Flibe energy \cite{sorensen_liquid-fluoride_2016}, 
etc.} would demonstrate similar benefits from the presents of the online gas 
removal system within nuclear island.


\section{Safety and operational parameters}
The significant changes of strong absorbers concentration in the fuel slightly 
shift the core spectrum which potentially might impact the reactor safety.
Rapid changes in fuel salt composition should not compromise critical safety 
margins.
I calculated major safety and operational parameters at various moments 
throughout the postulated transient using approaches from 
Sections~\ref{sec:safety-param} and \ref{ch5:saf_param}. 
The total temperature coefficient of reactivity ($\alpha_{ISO}$) must remain 
negative and the total control rod worth (CRW) must be sufficient to trip the 
reactor throughout the postulated transient. Ideally, we want to major safety 
and operational parameters stay almost constant because the changes in those 
parameter would require fast response from reactor control systems (i.e., the 
control rod jerk in response to the CRW change).

\subsection{Temperature coefficient of reactivity}
Figure~\ref{fig:msbr-lf-tc-evo} shows the temperature feedback coefficient 
dynamics for the \gls{MSBR} during the transient for various gas removal 
efficiencies ($\epsilon_{Xe}=0.536$ and 0.915). 
The Fuel Temperature Coefficient ($\alpha_{T,F}$) becomes less negative at the 
beginning of the transient for all cases. The reason for this is a 
slight spectrum hardening due to the $^{135}$Xe concentration peak changed the 
Doppler broadening of resonances. After that, the magnitude of $\alpha_{T,F}$ 
slowly inclined due to $^{135}$Xe removal from the fuel.
\begin{figure}[htbp!] % replace 't' with 'b' to 
	\centering
	\includegraphics[width=0.95\textwidth]{ch6/saf_par/tc_evo_kl25.png}\\
	\vspace{-12mm}
	\hspace{+0.05mm}
	\includegraphics[width=0.95\textwidth]{ch6/saf_par/tc_evo_kl100.png}
	\vspace{-3mm}
	\caption{Temperature feedback coefficients during the postulated transient 
	for the \gls{MSBR} operating with moderate ($\epsilon_{Xe}=0.536$, upper) 
	and high ($\epsilon_{Xe}=0.915$, lower) gas removal efficiency at the 
	\gls{BOL} (dashed line) and after 30 years of operation (solid line).
	The	uncertainty $\pm\sigma$ is shaded.}
	\label{fig:msbr-lf-tc-evo}
\end{figure}

The isothermal temperature coefficient, $\alpha_{ISO}$, is $-0.36\pm0.09$ 
$pcm/K$ at the beginning and remains stable during first 30 minutes of the 
transient for the moderate removal efficiency case.  Then, as the gas removal 
system reduces $^{135}$Xe concentration in the core, $\alpha_{ISO}$ becomes 
even more negative: $-1.52\pm0.09$ $pcm/K$ when the $^{135}$Xe mass stabilized 
at 1.5 g in about 5 hours after the shutdown. After power ramp-up from 0\% to 
100\%, $\alpha_{ISO}$ remains stable since the $^{135}$Xe mass increasing very 
slowly. On the whole, another interesting benefit from the online gas removal 
is improved passive safety (stronger temperature feedback coefficient) 
throughout and, possibly, a few days after the postulated transient due to low 
concentration of the $^{135}$Xe in the fuel salt.

For the high gas removal efficiency regime ($\epsilon_{Xe}=0.915$), the 
isothermal temperature coefficient worsens from $-0.54\pm0.09$ $pcm/K$ to 
approximately $-0.22\pm0.09$ $pcm/K$ during first 30 minutes after shutdown. 
Afterwards, when the gas removal system extracted major fraction of the 
$^{135}$Xe from the fuel salt, $\alpha_{ISO}$ became significantly more 
negative ($-1.39$ and $-1.56$ $pcm/K$ at the \gls{BOL} and after 30 years of 
operation, respectively) due to the spectrum softening. In brief, the 
temperature feedback in the \gls{MSBR} becomes stronger when neutron poisons 
concentration in the fuel decreases. As a result, flattening the $^{135}$Xe 
concentration peak improves the \gls{MSBR} passive safety.

Overall, the combination of fuel and moderator thermal feedback coefficients, 
$\alpha_{ISO}$, remains negative throughout the postulated transient. 
Moreover, simpler and cheaper gas removal system with extraction 
efficiency $\epsilon_{Xe}=0.536$ provided more predictable thermal feedback 
coefficient dynamics throughout the transient due to 
a more gradual change in the $^{135}$Xe concentration.

\subsection{Void coefficient of reactivity}
Figure~\ref{fig:msbr-lf-void-evo} demonstrates the void coefficient of 
reactivity evolution during the postulated transient. 
In contrast with the \gls{TAP} \gls{MSR}, the void coefficient of reactivity 
after 30 years of full-power operation is substantially greater than at the 
startup for both gas removal regimes. The reason for this is the \gls{MSBR} 
spectrum hardening toward \gls{EOL}, which is opposite to the \gls{TAP} 
\gls{MSR} spectrum evolution. Thus, an unexpected void insertion due to the 
gas separation system failure would lead to more severe consequences toward 
\gls{EOL}. 
\begin{figure}[htbp!] % replace 't' with 'b' to 
	\centering
	\includegraphics[width=0.92\textwidth]{ch6/saf_par/void_evo_kl25.png}\\
	\vspace{-12mm}
	\hspace{+0.05mm}
	\includegraphics[width=0.92\textwidth]{ch6/saf_par/void_evo_kl100.png}
	\vspace{-3mm}
	\caption{Void coefficient of reactivity as a function of time during 
	postulated transient
for the \gls{MSBR} operating with moderate 
	($\epsilon_{Xe}=0.536$, upper) and high ($\epsilon_{Xe}=0.915$, lower) gas 
	removal efficiency at the \gls{BOL} (dashed line) and after 30 years of 
	operation (solid line).}
	\label{fig:msbr-lf-void-evo}
\end{figure}

For the high gas removal efficiency, $\alpha_V$ fluctuates during the 
postulated transient between 42 and 61 $pcm/$void\% at the \gls{BOL} and 
between 87 and 102 $pcm/$void\% after 30 years of operation. The $^{135}$Xe 
concentration spike caused corresponding $\alpha_V$ drop due to short-term 
spectrum hardening. Then, $\alpha_V$ quickly recovers to its initial value. 
Similarly to the temperature feedback coefficient, the moderate gas removal 
efficiency provided more predictable $\alpha_V$ dynamics throughout the 
transient. Additionally, significantly smaller $\alpha_V$ raise toward 
\gls{EOL} for the case with $\epsilon_{Xe}=0.536$ ($\Delta\alpha_V\approx25$ 
$pcm/$void\%) will simplify the gas removal backup safety mechanism. 
Overall, all observed changes in the void coefficient of reactivity throughout 
the load-following transient for all cases are within 3-$\sigma$ range 
($\sigma_{\alpha_V}\pm5$ $pcm/$\%). This observations should be taken 
into account in the \gls{MSBR} accident analysis and safety
justification.

\subsection{Reactivity control rod worth}
Figure~\ref{fig:lf-msbr-crw-evo} shows the control rod worth evolution 
during the postulated transient. For the high gas removal efficiency regime 
with the \gls{EOL} fuel composition, the control rod worth dropped by $46\pm9$ 
$pcm$ during first 30 minutes after the shutdown due to short-term spectrum 
hardening related to the $^{135}$Xe concentration peak. In next 30 minutes, 
the CRW recovers to its initial value and keeps increasing throughout the 
transient because the gas removal system steadily reduces $^{135}$Xe 
concentration in the core. Notably, the control rod worth is greater at the 
\gls{BOL} because the $^{10}$B (used as absorber in the control rods) 
absorption cross section declines rapidly with energy. Overall, the control 
rod worth benefits from the \gls{MSBR} spectrum softening toward \gls{EOL}.

For the moderate gas separation efficiency regime, the control rod worth 
remains almost constant during first hour after shutdown for the both 
\gls{BOL} and \gls{EOL}. Afterwards, the CRW increased by 4\% due to the 
spectrum softening caused by the $^{135}$Xe concentration incline. As for 
other safety parameters, the control rod worth also benefits 
from less effective gas removal system due to smother xenon concentration 
dynamics, and, thus, more predictable neutron spectrum shift. 
\begin{figure}[htbp!] % replace 't' with 'b' to 
	\centering
	\includegraphics[width=0.95\textwidth]{ch6/saf_par/crw_evo_kl25.png}\\
	\vspace{-10mm}
	\hspace{+0.05mm}
	\includegraphics[width=0.95\textwidth]{ch6/saf_par/crw_evo_kl100.png}
	\vspace{-3mm}
	\caption{Total control rod worth as a function of time during 
		postulated transient
for the \gls{MSBR} operating with moderate 
		($\epsilon_{Xe}=0.536$, upper) and high ($\epsilon_{Xe}=0.915$, lower) 
		gas removal efficiency at the \gls{BOL} (dashed line) and after 30 
		years of operation (solid line).}
	\label{fig:msbr-lf-crw-evo}
\end{figure}

Unfortunately, the total control rod worth is insufficient to shut down the 
reactor throughout the postulated transient. The reactivity change during the 
transient is up to 2550 $pcm$ while the total control rod worth is only about 
$1250-1425$ $pcm$. The \gls{MSBR} was designed with only two graphite and two 
boron-carbide rods located in the center of the core (see 
Figure~\ref{fig:msbr_elev_view}) for operative reactivity control and relied 
heavily on fissile feed adjustment as a primary reactivity control 
mechanism. However, the fissile feed cannot be adjusted quickly and nuclear  
regulations required control rods have sufficient worth to safely shut down 
the reactor at any time. Therefore, the control rods design in the \gls{MSBR} 
must be reexamined to ensure the total control rod worth at least 3000 $pcm$ 
to ensure safety during the transient with rapid power change.


\section{Concluding remarks}
This chapter demonstrated SaltProc v1.0 capabilities to simulate the 
short-term depletion with the power variation from 0\% to 100\% for the 
\gls{MSBR}. I applied methodology from Chapter 5 to investigate the xenon 
poisoning effect in the \gls{MSBR} for three various gas removal system 
regimes: (1) no gas removal ($\epsilon_{Xe}=0.0$), (2) moderate gas removal 
efficiency ($\epsilon_{Xe}=0.536$), and (3) high gas removal efficiency 
($\epsilon_{Xe}=0.915$). 

When the gas removal system is inactive, $^{135}$Xe concentration peaked in 
about 7.5 hours after shutdown which caused the reactivity drop by 1457 and 
1035 $pcm$ for the startup and equilibrium fuel salt composition. Such 
negative effect of the xenon poisoning is consistent with other thermal 
reactor designs (i.e., -1500 $pcm$ for \gls{PWR} 
\cite{rykhlevskii_impact_2019}). In contrast with results for the \gls{TAP} 
\gls{MSR} in Chapter 5, the \gls{MSBR} demonstrated significant negative 
impact of the $^{135}$Xe concentration spike after shutdown on the core 
neutronics. The reason for that is significantly greater initial 
$^{135}$I/$^{135}$Xe concentration ratio: 2.45 and 1.0 for \gls{MSBR} and 
\gls{TAP} reactor at the \gls{BOL}, respectively. Thus, the $^{135}$Xe peak is 
significantly higher for the \gls{MSBR} than for \gls{TAP} reactor: +56\% 
and +0.33\%, respectively. Finally, the $^{135}$Xe parasitically absorbs 
substantially more neutrons in thermal (\gls{MSBR}) than in epithermal 
(\gls{TAP} \gls{MSR}) neutron spectrum which amplifies the xenon poisoning 
effect when the spectrum softens. In contrast with the spectrum thermalization 
toward \gls{EOL} in the \gls{TAP} reactor, in the \gls{MSBR} the neutron 
spectrum hardens toward \gls{EOL} due to plutonium and other strong absorbers 
accumulation in the fuel salt. Thus, for the \gls{MSBR} the xenon poisoning 
effect becomes less severe toward \gls{EOL}. 

The online gas removal in the \gls{MSBR} demonstrated impressive positive 
impact on the core neutronics. The gas removal system operation almost 
eliminated effect of xenon poisoning by removing vast majority of $^{135}$Xe 
during first hour after the shutdown. During the first 30-minutes interval, 
the reactivity dropped by 161 and 189 $pcm$ for moderate and high removal 
efficiency, respectively. Afterwards, the reactivity raised by approximately 
$+2700$ $pcm$ for both efficiencies in a few hours because the $^{135}$Xe mass 
in the fuel fell from 14 to 1-2 g. Indeed, the $^{135}$Xe loss due to decay 
and active gas removal greatly  overcame its only gain from the $^{135}$I 
decay (no fission happens, thus, no new $^{135}$I is produced). Notably, the 
amplitude of the reactivity swing after shutdown is larger for the \gls{BOL} 
when the xenon reactivity worth is greater due to softer neutron spectrum. 
Finally, significantly lower gas removal efficiency ($\epsilon_{Xe}=0.536$ 
instead of 0.915) provided comparable benefits to the \gls{MSBR} core 
neutronics during the postulated load-following transient.

Finally, this chapter demonstrated that the \gls{MSBR} maintains major safety 
margins throughout the postulated load-following transient. Thus, the 
temperature coefficient of reactivity and the total control rod worth worsen 
slightly during first 30 minute of the transient when the $^{135}$Xe 
concentration peaked causing corresponding neutron spectrum hardening. After 
that, the fast $^{135}$Xe concentration decline improved all safety and 
operational parameters among the cases. Unfortunately, the reactivity worth of 
two control rods made of boron carbide (B$_4$C) is insufficient to compensate 
huge reactivity change after shutdown. Even though the total control worth 
rises throughout the transient, the reactivity system design is unfeasible
for load-following and must be redesigned. 

In conclusion, the xenon poisoning effect impact on the \gls{MSBR} neutronics 
is much stronger than on the \gls{TAP} \gls{MSR}. Therefore, the \gls{MSBR} 
without 
gas removal system is incapable to reduce power from 100\% to 0\% and then 
restart anytime due to severe effect of xenon poisoning. However, the online 
gas removal even with moderate separation efficiency (e.g., 
$\epsilon_{Xe}=0.536$) aids to eliminate the iodine pit problem and enable 
load-following capability of the \gls{MSBR} without compromising its safety. 
For the best results, the gas removal system must have smart control system 
coupled with reactivity control and power regulation systems. Ideally, the 
separation efficiency should be boosted right before the power ramp down and 
during first few moments after power drop to flatten the $^{135}$Xe peak. 
Afterwards, the control system would reduce the removal efficiency to avoid a
vast positive reactivity insertion due to fast $^{135}$Xe concentration cut 
down. Overall, more detailed study of power changing transients must be 
performed using SaltProc v1.0 with better time resolution (i.e., a 1-min 
interval) to better understand how to adjust the gas removal efficiency during 
load-following.
