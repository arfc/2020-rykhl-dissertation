\chapter{Error propagation in depletion calculations}
In the \gls{MC} depletion analyses, the uncertainties on predicted isotopic 
composition are caused by two primary factors: stochastic uncertainty in 
the computed flux and uncertainty in the nuclear data (cross sections, fission 
yields, decay constants). In \gls{MC} reactor physics software, the stochastic 
uncertainty of single burnup step is superposed with errors, propagated 
throughout calculation from previous steps. Over time, these errors 
accumulate, and cumulative error in the predicted number density might be 
significant for the lifetime-long fuel depletion calculations.

Takeda \emph{et al.} first proposed a method to evaluate uncertainty of the 
number density in the \gls{MC} simulations applying the sensitivities of the 
burnup matrix to cross sections and number densities. Takeda and colleagues 
propagated covariances of the cross section library and obtained number 
density uncertainty due to the cross section error of about 4\% for major 
heavy isotopes ($^{235}$U, $^{239}$Pu, $^{241}$Pu) after 400-day 
\gls{MC} burnup calculation for homogeneous model of fast reactor. Notably, 
the number density uncertainty due to the stochastic error in the \gls{MC} was 
much lower: about 0.03\% for $^{241}$Pu, 0.02\% for $^{235}$U, and $<0.004$\% 
for $^{238}$U. The Takeda model showed that statistical error contribution to 
the total error in number densities of major heavy isotopes and \glspl{FP} is 
less than 1\% \cite{takeda_estimation_1999}. Finally, substantial neutron 
history ($N$) increase can theoretically reduce the stochastic error to zero 
but it is enormously expensive due to slow convergence ($O(\sqrt{N}$) of Monte 
Carlo method.

In a similar vein, Rochman \emph{et al.} used Serpent Monte Carlo burnup code 
and in-house sampler to analyze total uncertainty for actinide number density 
for typical \gls{PWR} fuel assembly model. Rochman propagated nuclear data 
uncertainties by repeating thousand times the same burnup calculation, each 
time using a different file with randomly perturbed nuclear data.


\section{Statistical uncertainty in depleted fuel composition}

\section{Nuclear data related uncertainty in depleted fuel composition}

\section{Concluding remarks}