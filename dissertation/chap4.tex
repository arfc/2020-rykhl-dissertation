\chapter{Tool demonstration: Transatomic Power MSR}

This chapter demonstrate SaltProc v1.0 capabilities for the \gls{TAP} 
\gls{MSR}. The \gls{TAP} concept was selected because it is well analyzed in 
the literature \cite{betzler_two-dimensional_2017, betzler_assessment_2017-1};
thus, code-to-code verification with ChemTRITON/SCALE is possible 
\cite{betzler_assessment_2017-1}. The demonstration was performed for two 
timescales:
\paragraph*{Long-term.} I performed the \gls{TAP} \gls{MSR} core lifetime-long 
(25 years) depletion simulation with moderate time resolution (3-day depletion 
step) and constant, 100\% power level. The 
results obtained with SaltProc v1.0 are compared with recent efforts discussed 
in Chapter 1, more specifically with full-core \gls{TAP} depletion analysis 
with ideal removal efficiencies by Betzler \emph{et al.}  
\cite{betzler_assessment_2017-1}. This validation effort showed that SaltProc 
v1.0 solution is correct for the case with \emph{ideal extraction efficiency}.
\paragraph*{Short-term (transient).} I performed the 24-hour-long depletion 
simulation with changing, load following reactor power with the fine time 
resolution (variable depletion step from 1 to 30-min). The depletion 
calculation for the \gls{TAP} in load following regime capture the effects of 
xenon poisoning and evaluate the benefit of using an online gas removal system.

In this chapter, I also analyze the reactor load-following capability for 
various moderator configurations and fuel salt compositions to bound design 
parameters of gas removal system to ensure load-following operation. 

\section{Transatomic Power Molten Salt Reactor design description}

The \gls{TAP} concept is a 1250 MW$_{th}$ \gls{MSR} with a LiF-based uranium 
fuel salt \cite{transatomic_power_corporation_technical_2016}. This concept 
uses configurable zirconium hydride rods as the moderator while most \gls{MSR} 
designs typically propose high-density reactor graphite. Zirconium hydride 
offers a much higher neutron moderating density than graphite: much less 
zirconium hydride volume is needed to achieve a thermal energy spectrum 
similar to one obtained with graphite moderator. Moreover, zirconium hydride 
has a much longer lifespan in extreme operational conditions (high 
temperature, large neutron flux, chemically aggressive salt) than reactor 
graphite. Finally, zirconium hydride is a nonporous material and holds up 
fewer neutron poisons (e.g., xenon, krypton) than does high-density 
reactor graphite.

In this section, the design characteristics and reprocessing plant design are 
based on information presented in the TAP white papers  
\cite{transatomic_power_corporation_technical_2016,
transatomic_power_corporation_neutronics_2016} and \gls{ORNL} technical 
reports \cite{betzler_two-dimensional_2017, betzler_assessment_2017-1}.

\subsection{General design description}

Figure~\ref{fig:tap-rendering} demonstrates a rendering of the primary and 
secondary loop of the \gls{TAP} \gls{MSR} seated inside a concrete nuclear 
island. Figure~\ref{fig:tap-primary-scheme} shows the schematic design of a 
520 MW$_{e}$, 2-loop nuclear reactor system with an intermediate salt loop.
\begin{figure}[h] % replace 't' with 'b' to 
	\centering
	\includegraphics[width=\textwidth]{ch4/tap_render.jpg}
	\caption{Rendering of the \gls{TAP} \gls{MSR}. The fission happens in the 
		fuel salt inside the reactor vessel (1). The heat generated by 
		self-sustaining nuclear fission reaction would be transferred to the 
		secondary salt by heat exchanger (2), which would boil water in the 
		steam 
		generator (3). Valves made of salt with higher melting point (4) would 
		melt in case of emergency, allowing the salt to drain into a drain 
		tank 
		(5) which is able to passively dissipate decay heat	(reproduced from 
		\cite{strickland_transatomic_2014}, illustration by Emily Cooper).}
	\label{fig:tap-rendering}
\end{figure}
\begin{figure}[h] % replace 't' with 'b' to 
	\centering
	\includegraphics[width=\textwidth]{ch4/tap_simplified_scheme.png}
	\caption{Simplified schematic of the \gls{TAP} \gls{MSR} primary and  
		secondary loops (reproduced from the Transatomic Power Technical White 
		Paper \cite{transatomic_power_corporation_technical_2016}). Figure 
		legend: 
		A) reactor vessel, b) fuel salt pumps, C) primary heat exchangers, D) 
		freeze plug, E) primary loop drain tank, F) secondary loop salt pump, 
		G) 
		steam generator, H) secondary loop drain tank, I) fuel catch basin.}
	\label{fig:tap-primary-scheme}
\end{figure}

The \gls{TAP} design (figure~\ref{fig:tap-side-view}) is very similar to 
original \gls{MSRE} design developed by \gls{ORNL} 
\cite{haubenreich_experience_1970} but has two major innovations: 
the fuel salt composition and the moderator. The \gls{MSRE}'s 
LiF-BeF$_2$-ZrF$_4$-UF$_4$ salt has been substituted with LiF-UF$_4$ salt 
which allows for an increase in the uranium concentration within the fuel salt 
from 0.9 to 27.5\% while maintaining a relatively low melting point 
(490$^{\circ}$C compared with 434$^{\circ}$C for the original \gls{MSRE}'s 
salt) \cite{betzler_two-dimensional_2017}. The graphite has a very high 
thermal scattering cross section which makes it a perfect moderator but has 
a few major drawbacks: 
\begin{enumerate}[label=(\alph*), itemsep=-1ex]
	\item low lethargy gain per collision requires a large volume of 
	moderator to be present to reach criticality, which leads to a larger core 
	and obstructs the core power density;
	\item even special reactor-grade graphite has relatively high porosity, 
	consequently, it holds gaseous \glspl{FP} (e.g., tritium, xenon) in pores;
	\item reactor graphite lifespan in a commercial reactor is 
	approximately 10 years \cite{robertson_conceptual_1971}.
\end{enumerate}
As previously mentioned, to resolve these issues, the \gls{TAP} concept uses 
zirconium hydride instead of graphite, allowing for a more compact core and a 
significant increase in power density. These two innovative design choices, 
together with a configurable moderator (the moderator-to-fuel ratio can be 
changed during operation), facilitate the deployment of this conceptual design 
in the current commercially available 5\% enriched \gls{LEU} fuel cycle. 
\begin{figure}[h] % replace 't' with 'b' to 
	\hspace{+2.2in}
	\includegraphics[width=0.65\textwidth]{ch4/tap_front_view.png}
	\caption{The \gls{TAP} \gls{MSR} schematic view showing movable moderator 
		rod bundles and shutdown rod (reproduced from Transatomic Power 
		White Paper \cite{transatomic_power_corporation_technical_2016}).}
	\label{fig:tap-side-view}
\end{figure}

The \gls{TAP} \gls{MSR} primary loop contains the reactor core volume  
(including the zirconium hydride moderator rods with silicon carbide  
cladding), pumps, and primary heat exchangers. Pumps circulate the  
LiF-(Act)F$_4$ fuel salt through the primary loop. The pumps, vessels, tanks, 
and piping are made of a nickel-based alloy (similar to Hastelloy-N\footnote{ 
Hastelloy-N is very common in \gls{MSR} designs now, but was developed at 
\gls{ORNL} in the \gls{MSRE} program that started in the 1950s.}), which 
is highly resistant to corrosion in various molten salt environments. Inside 
the reactor vessel, near the zirconium hydride moderator rods, the fuel salt 
is in a critical configuration and generates heat. Table~\ref{tab:tap_tab} 
contains details of the \gls{TAP} system design which are taken from technical 
white paper \cite{transatomic_power_corporation_technical_2016} and a 
neutronics overview \cite{transatomic_power_corporation_neutronics_2016} as 
well as \gls{ORNL} analysis of the \gls{TAP} design 
\cite{betzler_two-dimensional_2017, betzler_assessment_2017-1}. 
%%%%%%%%%%%%%%%%%%%%%%%%%%%%%%%%%%%%%%%%
\begin{table}[h!]
	\caption{Summary of principal data for the \gls{TAP} \gls{MSR} 
		(reproduced from \cite{betzler_assessment_2017-1, 
		transatomic_power_corporation_technical_2016}). }
		\centering
	\begin{tabularx}{0.8\textwidth}{L R}
		\hline
		Thermal power   		& 1250 MW$_{th}  $       		\\ 
		Electric power		    & 520 MW$_e  $ 			 		\\ 
		Gross thermal efficiency& 44\%     				 		\\  
		Outlet temperature      & 620$^{\circ}$C         		\\ 
		Fuel salt components    & LiF-UF$_4$				    \\  
		Fuel salt composition   & 72.5-27.5 mole\%				\\  
		Uranium enrichment      & 5\% $^{235}$U          	    \\
		Moderator               & Zirconium Hydride (ZrH$_{1.66}$) rods \\
								& (with silicon carbide cladding)       \\
		Neutron spectrum at the \gls{BOL} & intermediate \\
		\qquad\qquad\qquad\qquad\space at the \gls{EOL}     & thermal      \\
		\hline
	\end{tabularx}
	\label{tab:tap_tab}
\end{table}
%%%%%%%%%%%%%%%%%%%%%%%%%%%%%%%%%%%%%%%%%%%%%%%%%%%%%%%%%%%%%%%%%%%%%%%%%%%%%%%

\subsection{Reactor core design}
In the \gls{TAP} core (Figure~\ref{fig:tap-core-ben}), fuel salt flows around 
moderator assemblies consisting of lattices of zirconium hydride rods clad in 
a corrosion-resistant silicone carbide. The \gls{TAP} reactor pressure vessel 
is a cylinder with an inner radius 150 cm, height 350 cm, and wall thickness 5 
cm made of a nickel-based alloy. 
\begin{figure}[t] % replace 't' with 'b' to 
	\includegraphics[width=\textwidth]{ch4/tap_core_ornl.png}
	\vspace{-0.35in}
	\caption{The \gls{TAP} \gls{MSR} schematic core view showing moderator 
		rods (reproduced from ORNL/TM-2017/475  
		\cite{betzler_assessment_2017-1}).}
	\label{fig:tap-core-ben}
\end{figure}

The \gls{SVF} in the core is parameter similar to wide-used moderator-to-fuel 
ratio and can be defined as:
\begin{align}
SVF &= \frac{V_F}{V_F+V_M} = \frac{1}{1+V_M/V_F}
\intertext{where}
V_F &= \mbox{the fuel volume $[m^3]$} \nonumber \\
V_M &= \mbox{the moderator volume $[m^3]$} \nonumber \\
V_M/V_F &= \mbox{the moderator-to-fuel salt ratio $[-]$.} \nonumber
\end{align}

Figure~\ref{fig:svf-predetermined} shows the \gls{SVF} variation during  
operation that shifts the reactor neutron energy spectrum from intermediate to 
thermal to maximize fuel burnup. At the \gls{BOL}, a high \gls{SVF} is 
selected to obtain relatively hard spectrum and enhance fertile material 
($^{238}$U) conversion into the fissile material ($^{239}$Pu) when the 
startup fissile material ($^{235}$U) inventory is still large. As fissile 
concentration in the fuel salt declines, an additional moderator are 
introduced to maintain criticality, leading to salt volume fraction decrease 
(see Figure~\ref{fig:svf-predetermined}).

Initial \gls{TAP} concept suggested vary salt volume fraction by inserting 
fixed-sized moderator rods via the bottom of the reactor vessel (for safety 
considerations), similarly to moving the control rods in a \gls{BWR}, as shown 
in Figure~\ref{fig:tap-side-view} 
\cite{transatomic_power_corporation_neutronics_2016}. The later \gls{TAP} 
concept proposes reducing the \gls{SVF} by reconfiguring the moderator rods 
during regular shutdown for reactor maintenance 
\cite{betzler_assessment_2017-1}. For the 
\gls{TAP} reactor, \gls{EOL} occurs when the maximum number of moderator rods 
are inserted into the core, and a further injection of fresh fuel salt does 
not alter criticality. Unmoderated salt is flowing in the annulus between the 
core, and the vessel wall provides for a potential reduction in fast neutron 
flux at the vessel structural material .
\begin{figure}[t] % replace 't' with 'b' to 
	\includegraphics[width=\textwidth]{ch4/svf_predetermined.png}
	\caption{The change in SVF as a function of burnup in the \gls{TAP} 
	reactor (reproduced from Transatomic Power Neutronics Overview  
	\cite{transatomic_power_corporation_neutronics_2016}).}
	\label{fig:svf-predetermined}
\end{figure}


\subsection{Fuel salt reprocessing system}
The \gls{TAP} nuclear island contains a fission product removal system. 
Gaseous \glspl{FP} are continuously removed using an off-gas system while 
liquid and 
solid \glspl{FP} are extracted via a chemical processing system. As these 
byproducts are gradually removed, a small quantity of fresh fuel salt is 
regularly added to the primary loop. This process conserves a constant fuel 
salt mass and keeps the reactor critical. In contrast with the \gls{MSBR} 
reprocessing system, the \gls{TAP} design does not need a protactinium 
separation and isolation system because it operates in a uranium-based 
single-stage fuel cycle. The authors of the \gls{TAP} concept suggested three 
distinct fission product removal methods 
\cite{transatomic_power_corporation_neutronics_2016}:
\paragraph*{Off-Gas System:} The off-gas system removes gaseous fission 
products such as krypton and xenon, which are then compressed and stored 
temporarily until they have decayed to the background radiation level. Trace 
amounts of tritium are also removed and bottled in a liquid form via the same 
process. Also, the off-gas system directly removes a small fraction of the 
noble metals.
\paragraph*{Metal Plate-Out/Filtration:} A nickel mesh filter removes noble 
and semi-noble metal solid fission products as they plate out onto internal 
surface of the filter.
\paragraph*{Liquid Metal Extraction:} Lanthanides and other non-noble metals 
stay dissolved in the fuel salt. They generally have a lower capture cross 
section and thus absorb fewer neutrons than $^{135}$Xe, but their extraction 
is essential to ensure normal operation. In the \gls{TAP} reactor, lanthanide 
removal is accomplished via a liquid-metal/molten salt extraction process 
similar to that developed for \gls{MSBR} by \gls{ORNL}  
\cite{robertson_conceptual_1971}. The process converts the dissolved 
lanthanides into a well-understood oxide waste form, similar to that of 
\gls{LWR} \gls{SNF}. This oxide waste comes out of the \gls{TAP} reprocessing 
plant in ceramic granules and can be sintered into another convenient form for 
storage \cite{transatomic_power_corporation_technical_2016}.

Figure~\ref{fig:tap-reproc} shows the principal design of the \gls{TAP} 
primary loop, including an off-gas system, nickel mesh filter, and lanthanide 
chemical extraction facility. Similarly to the \gls{MSBR}, an off-gas system 
is also based on a simple process of helium sparging through fuel salt with 
consequent gas bubbles removed before returning the fuel salt to the core (see 
Section~\ref{sec:gas-separ}). 
Nevertheless, one crucial difference must be noted: the \gls{MSBR} gas 
separation system suggested helium injection and subsequent transport of the 
voids throughout the primary loop, including the core for at least ten full 
loops \cite{robertson_conceptual_1971}. 
\begin{figure}[htp!] % replace 't' with 'b' to 
	\centering
	\includegraphics[width=\textwidth]{ch4/tap_primary_loop.png}
	\caption{Simplified \gls{TAP} primary loop design including off-gas system 
		(blue), 
		nickel filter (orange) and liquid metal extraction system (green) 
		(reproduced from \cite{transatomic_power_transatomic_2019}).}
	\label{fig:tap-reproc}
\end{figure}

Introduction of the void (helium bubbles) during operation is a significant 
concern for safe, stable operation because the increase of void fraction in 
the fuel salt when it enters back to the core would cause unpredictable 
reactivity change. Kedl stated, ``Average loop void fractions as high as 1\% 
are undesirable... it is desirable to keep the average loop void fraction well 
below 1\%.''\cite{robertson_conceptual_1971} but he did not offer an 
explanation why. In fact, the \gls{MSBR} design targeted 0.2\% average void in 
the fuel salt \cite{robertson_conceptual_1971} and successfully operated the 
\gls{MSRE} with average void fraction below 0.7\% \cite{compere_fission_1975}.
We can reduce void fraction in the fuel salt to negligible levels by using an 
effective gas separator for stripping helium/xenon bubbles before returning 
the salt to a primary loop (Figure~\ref{fig:tap-reproc}, blue block). 

Noble and semi-noble metal solid fission products tend to plate out onto metal 
surfaces including piping, heat exchanger tubes, reactor vessel inner surface, 
etc. Previous research by \gls{ORNL} \cite{robertson_conceptual_1971} reported 
that about 50\% of noble and semi-noble metals would plate out inside 
\gls{MSBR} systems (including off-gas system) without any special treatment. 
To improve the extraction efficiency of these fission products, the \gls{TAP} 
concept suggested employing a nickel mesh filter located in a bypass stream in 
the primary loop (Figure~\ref{fig:tap-reproc}, orange block). The main idea of 
this filter is to create large nickel surface area using porous metal (e.g., 
Inconel fibers). The fuel salt is flowing throughout the filter and noble 
metals plate-out on the internal filter surface. 

This Liquid Metal Extraction process for the \gls{TAP} concept has been 
adopted from the \gls{MSBR}. The \gls{MSRE} demonstrated a liquid-liquid 
extraction process for removing rare earths and lanthanides from fuel salt and 
estimated efficiency of this process. 
In fact, due to similarities in reprocessing schemes, the \gls{TAP} project 
reported almost the same set of elements for removal and similar effective 
cycle times as suggested for \gls{MSBR} (Table~\ref{tab:reprocessing_list}). 
The \gls{TAP} neutronics whitepaper specifies additional low-probability 
fission products and gases that should be removed during operation. These 
elements are categorized into the previously defined processing groups, but 
the removal rates of most of these elements (all except for hydrogen) 
are very low.

Details of gas removal and fuel reprocessing systems have historically 
been conceptual. Accordingly, liquid-fueled system design including the 
\gls{TAP} concept usually assumes ideal (rather than realistically 
constrained) removal efficiencies for reactor performance simulations. In this 
thesis, I developed a realistic online reprocessing system and reactor model 
to capture the dynamics of fuel composition evolution during reactor 
operation. Gas removal efficiency is variable in that model, described using 
mathematical correlation from Chapter 2 (see Equation~\ref{eq:gas_eff}). For 
the other \glspl{FP}, a fixed\footnote{ 
	Published information about dynamics of extraction efficiency during 
	reactor 
	operation for seminoble metals, volatile fluorides, and rare earths is 
	insufficient to inform a variable removal efficiency.}, non-ideal 
extraction efficiency based on cycle time from  
Table~\ref{tab:reprocessing_list} was used to inform the fuel reprocessing 
model.

%%%%%%%%%%%%%%%%%%%%%%%%%%%%%%%%%%%%%%%%
\begin{table}[htp!]
	\centering
	\caption{The effective cycle times for fission products removal  from the 
		\gls{TAP} reactor (reproduced from \cite{betzler_implementation_2017} 
		and 
		\cite{transatomic_power_corporation_neutronics_2016}).}
	\begin{tabular}{p{0.2\textwidth} p{0.42\textwidth} p{0.12\textwidth} 
			p{0.14\textwidth}}
		\hline 
		%\begin{tabularx}{\linewidth}{l X} \toprule 
		\textbf{Processing group} & \qquad\qquad\qquad \textbf{Nuclides} & 
		\textbf{Removal Rate (s$^{-1}$)} & \textbf{Cycle time (at full power)} 
		\\ [5pt] \hline 
		\multicolumn{3}{c}{\textit{Elements removed in the \gls{MSBR} concept 
		and adopted for the \gls{TAP}} \cite{robertson_conceptual_1971}} \\
		Volatile gases & Xe, Kr								  & 5.00E-2 & 20 
		sec \\ [5pt]
		Noble metals & Se, Nb, Mo, Tc, Ru, Rh, Pd, Ag, Sb, Te & 5.00E-2 & 20 
		sec \\ [5pt]
		Seminoble metals & Zr, Cd, In, Sn	  				  & 5.79E-8 & 200 
		days \\ [5pt]
		Volatile fluorides & Br, I 							  & 1.93E-7 & 60 
		days \\ [5pt]
		Rare earths & Y, La, Ce, Pr, Nd, Pm, Sm, Gd           & 2.31E-7 & 50 
		days \\ [5pt]
		\qquad & Eu & 2.32E-8 & 500 days \\ [5pt]
		Discard & Rb, Sr, Cs, Ba & 3.37E-9 & 3435 days \\ [5pt] 
		\hline
		
		\multicolumn{3}{c}{\textit{Additional elements removed} 
			\cite{transatomic_power_corporation_neutronics_2016, 
				betzler_implementation_2017}  } \\
		Volatile gases & H								  	& 5.00E-2 & 20 
		sec    \\ [5pt]
		Noble metals & Ti, V, Cr, Cu						& 3.37E-9 & 3435 
		days \\ [5pt]
		Seminoble metals & Mn, Fe, Co, Ni, Zn, Ga, Ge, As   & 3.37E-9 & 3435 
		days \\ [5pt]
		Rare earths & Sc									& 3.37E-9 & 3435 
		days \\ [5pt]
		Discard & Ca										& 3.37E-9 & 3435 
		days \\ [5pt] 
		\hline
	\end{tabular}
	\label{tab:reprocessing_list}
	\vspace{-0.9em}
\end{table}
%%%%%%%%%%%%%%%%%%%%%%%%%%%%%%%%%%%%%%%%%%%%%%%%%%%%%%%%%%%%%%%%%%%%%%%%%%%%%%%


\section{TAP system model}
In this section, the \gls{TAP} core and fuel salt reprocessing system models  
for demonstrating SaltProc v1.0 are described in details. I used these models 
for SaltProc demonstration and validation for both long-term and short-term 
cases.

\subsection{Serpent 2 full-core model} \label{sec:tap_model}
Nest and lattice geometry types as well as universe transformation 
capabilities of Serpent \cite{leppanen_serpent_2014} are employed to 
represent \gls{TAP} core. Figure~\ref{fig:tap-serpent-plan} shows the $XY$ 
section of the whole-core model at the expected reactor operational level 
when all control rods are fully withdrawn. Figures~\ref{fig:tap-serpent-elev} 
and \ref{fig:tap-serpent-elev-zoom} show a longitudinal section of the 
reactor. This model contains the moderator rods with silicon carbide cladding, 
pressure vessel, and inlet and outlet plena (Table~\ref{tab:tap_model_param}). 
Fuel salt flows around rectangular moderator assemblies consisting of lattices 
of small-diameter zirconium hydride rods in a corrosion-resistant material. 
The salt volume fraction for Figure ~\ref{fig:tap-serpent-plan} is 0.917204, 
which means the modeled core is under-moderated and has intermediate neutron 
spectrum. Quarter-core configurations of the \gls{TAP} core with various salt 
volume fraction, used in the current work to maintain criticality for 
reasonable operational period ($>20$ years), are listed in 
Table~\ref{tab:tap_adjustable_core}, Figures~\ref{fig:tap-406-681}, and 
\ref{fig:tap-840-1668} in Appendix~\ref{appex:geometries}.
\begin{figure}[htp!] % replace 't' with 'b' to 
	\centering
	\includegraphics[width=0.75\textwidth]{ch4/tap_plan_view_serpent_347.png}
	\caption{An $XY$ section of the \gls{TAP} model at horizontal midplane 
		with fully withdrawn control rods at \gls{BOL} (347 moderator rods, 
		salt volume fraction is equal 0.917204). 
		The violet color represents zirconium hydride, and the yellow 
		represents fuel salt. 
		The blue color shows Hastelloy-N, a material used for the vessel wall, 
		and the pink color is the air.}
	\label{fig:tap-serpent-plan}
\end{figure}

\begin{figure}[htp!] % replace 't' with 'b' to 
	\centering
	\includegraphics[width=0.6\textwidth]{ch4/tap_elev_view_serpent_347.png}
	\caption{45$^{\circ}$ $XZ$ section of the \gls{TAP} core model.}
	\label{fig:tap-serpent-elev}
\end{figure}

To represent the reactivity control system, the model has: 
\begin{enumerate}[label=(\alph*), noitemsep]
	\item control rod guide tubes made of nickel-based alloy;
	\item control rods represented as a Boron Carbide (B$_4$C) cylinders 
	with a thin Hastelloy-N coating;
	\item air inside guide tubes and control rods;
\end{enumerate}
The control rods must be able to suppress excess reactivity at the \gls{BOL} 
when the core configuration is the most reactive and the neutron spectrum is 
the hardest. The control rod design shown on 
Figures~\ref{fig:tap-serpent-plan} and \ref{fig:tap-serpent-elev-zoom} is 
comprised of a cluster of 25 rods that provide a total reactivity worth of 
$4226\pm9pcm$ at the \gls{BOL}.

\begin{figure}[hbp!] % replace 't' with 'b' to 
	\centering
	\includegraphics[width=0.55\textwidth]{ch4/tap_elev_view_zoomed_serpent.png}
	\caption{Zoomed $XZ$ section of the top of the moderator and control rods  
		in the \gls{TAP} model. The orange color shows Boron Carbide 
		(B$_4$C) absorbers used for control rods.}
	\label{fig:tap-serpent-elev-zoom}
\end{figure}
%%%%%%%%%%%%%%%%%%%%%%%%%%%%%%%%%%%%%%%%%%%%%%%%%%
\begin{table}[ht!]
	\caption{Geometric parameters for the full-core 3D model of the 
		\gls{TAP} (reproduced from Betzler \emph{et al.} 
		\cite{betzler_assessment_2017-1}). }
	\centering
	\begin{tabularx}{0.9\textwidth}{s s x p{0.14\textwidth}}
		\hline
		\textbf{Component}&\textbf{Parameter}&\textbf{Value}& \textbf{Unit}   
		\\ \hline
		\multirow{4}{*}{\begin{tabular}[c]{@{}l@{}}Moderator\\ 
				rod\end{tabular}} 
		& Cladding thickness      	  			    & 0.10 & cm				 
		\\  
		& Radius 				      	  			& 1.15 & cm				 
		\\  
		& Length				      	  			& 3.0  & m				 
		\\  
		& Pitch				      	  			& 3.0  & cm  			 \\ 
		\hline 
		
		\multirow{2}{*}{\begin{tabular}[c]{@{}l@{}}Moderator\\ 
				assembly\end{tabular}} 
		& Array				      	  			& 5 $\times$ 5 & 
		rods$\times$rods \\  
		& Pitch				      	  			& 15.0 & cm    				 
		\\  \hline
		
		\multirow{4}{*}{\begin{tabular}[c]{@{}l@{}}Core\end{tabular}}          
		& Assemblies  				   	  			& 268  & assemblies/core 
		\\  
		& Inner radius			      	  			& 1.5  & 
		m    				 \\  
		& Plenum height			   	  			& 25.0 & cm    				 
		\\  
		& Vessel wall thickness     	  			& 5.0 & 
		cm    				 \\ \hline            
	\end{tabularx}
	\label{tab:tap_model_param}
\end{table}
%%%%%%%%%%%%%%%%%%%%%%%%%%%%%%%%%%%%%%%%%%%%%%%%

The control rod cluster is modeled using the \textbf{TRANS} Serpent 2 feature, 
which allows the user to easily change the control rod position during the 
simulation. The current works assumed that all control rods are fully 
withdrawn from the core (Figure~\ref{fig:tap-serpent-elev-zoom}), but user can 
use SaltProc v1.0 reactivity control capabilities to change control 
rod position during operation. In this dissertation, all figures of the core 
were generated using the built-in Serpent plotter.

The neutron population per cycle and the number of active/inactive cycles were 
chosen
to obtain a balance between minimizing uncertainty for a transport 
problem (28 pcm for
$k_{eff}$) and simultaneously minimizing computational 
time.


\subsection{Model of the fuel reprocessing system}
I thoroughly analyzed the original \gls{TAP} reprocessing system design 
(Figure~\ref{fig:tap-reproc}) and neutron poisons removal rates  
(Table~\ref{tab:reprocessing_list}) to determine a suitable reprocessing 
scheme for SaltProc v1.0 demonstration (Figure~\ref{fig:demo-repro-scheme}). 
This demonstration case assumed fixed, non-ideal ($<100$\%) removal 
efficiencies based on physical model for noble gas extraction efficiency 
discussed in Section~\ref{sec:gas-separ}. 
\begin{figure}[htp!] % replace 't' with 'b' to 
	\centering
	\includegraphics[width=\textwidth]{ch4/tap_saltproc_repro_scheme.png}
	\caption{\gls{TAP} reprocessing scheme flowchart used for demonstration of 
	SaltProc v1.0. Arrows represent material flow; percents represent fraction 
	of total mass flow rates; ellipses represent fuel reprocessing 
	system components; diamonds represent waste streams; the box shows 
		refuel material flow.}
	\label{fig:demo-repro-scheme}
\end{figure}

The gas removal components (sparger/contactor and entrainment separator) are 
located in-line because estimated full loop time for the fuel salt is about 20 
seconds and approximately equal to the cycle time 
(Table~\ref{tab:reprocessing_list}). To remove all volatile gases every 20 
seconds, the gas removal system must operate with 100\% of the core throughout 
flow rate (in-line gas removal system). For the demonstration case herein 
efficiency of xenon, krypton, and hydrogen extraction are determined using 
Peebles \emph{et al.} model (Equation~\ref{eq:gas_eff}) discussed earlier. 
Gas-liquid interfacial area per unit volume ($a$) to inform 
equation~\ref{eq:gas_eff} is a function of salt/gas flow rates and diameter of 
the gas bubbles \cite{sada_gas-liquid_1987}:
\begin{align}\label{eq:interfacial-area}
& a = \frac{6}{d_b}\frac{Q_{He}}{Q_{He} + Q_{salt}}
\intertext{where}
Q_{salt}&= \mbox{volumetric salt flow rate $[m^3/s]$} \nonumber \\
Q_{He}&= \mbox{volumetric helium flow rate $[m^3/s]$} \nonumber \\
d_b &= \mbox{helium bubble diameter $[m]$.} \nonumber
\end{align}

Extraction efficiencies are different for different gases because they have 
different solubility (Henry's law constant) in the salt. 
Table~\ref{tab:gas_removal_efficiency} reports Henry's law constants and 
corresponding efficiencies of migration noble gas to the helium bubbles in the 
sparger. Total separation efficiency (Table~\ref{tab:gas_removal_efficiency}) 
refers efficiency of extraction target gaseous element after performing helium 
sparging in the sparger following by separation of noble-gas-reach bubbles 
from the salt in the axial-flow centrifugal bubble separator 
\cite{gabbard_development_1974}.
%%%%%%%%%%%%%%%%%%%%%%%%%%%%%%%%%%%%%%%%
\begin{table}[htp!]
	\centering
	\caption{The noble gas extraction efficiency at working temperature 
	T=627$^{\circ}$C calculated using Peebles \emph{et al.} model 
	(Equation~\ref{eq:gas_eff}) 
	assuming liquid phase mass transfer coefficient 
	$K_L=0.0085$ $m/s$ \cite{peebles_removal_1968}, salt volumetric flow rate 
	$Q_{salt}=2$ $m^3/s$, helium volumetric flow rate $Q_{He}=0.1$ $m^3/s$, 
	helium bubbles diameter $d_b=0.508$ $mm$, and volume of the sparger 
	$V=1.4$ $m^3$.}
	\begin{tabularx}{\textwidth}{L R R R}
		\hline
		\textbf{Element}  & \textbf{Dimensionless} & 
		\multicolumn{2}{c}{\textbf{Efficiency of}}\\
		& \textbf{Henry's law constant at T=700$^{\circ}$C} & 
		\textbf{migration to He bubbles}&\textbf{total separation$^{\star}$}\\ 
		\hline
		Xe &$0.057\times 10^{-3}$ 
		\cite{blander_solubility_1959}&0.9630&0.9149\\
		Kr &$\approx0.283\times 10^{-3}$ \cite{blander_solubility_1959}&0.9595 
		&0.9115\\
		H  &$3.870\times 10^{-3}$ \cite{tomkins_gases_2016}&0.9066&0.8613\\
		\hline
	\end{tabularx}
	\begin{tablenotes}
		\small
		\item$^{\star}$With axial-flow centrifugal bubble separator by 
		Gabbard \emph{et al.} which allows the 
		bubble separation efficiency 95\% \cite{gabbard_development_1974}.
	\end{tablenotes}
	\label{tab:gas_removal_efficiency}
	\vspace{-0.9em}
\end{table}
%%%%%%%%%%%%%%%%%%%%%%%%%%%%%%%%%%%%%%%%%%%%%%%%%%%%%%%%%%%%%%%%%%%%%%%%%%%%%%%

The nickel filter in the \gls{TAP} concept is designed to extract 
noble/semi-noble metals and volatile fluorides 
(Table~\ref{tab:reprocessing_list}). Similarly to volatile gases, 
noble metals must be removed every 20 seconds and, hence, the filter should 
operate in-line also.The nickel filter removes a wide range of elements with 
various efficiencies (Table~\ref{tab:reprocessing_list}).

Lanthanides and other non-noble metals generally have a lower capture  
cross section and absorb fewer neutrons than gases and noble metals. These 
elements can be removed via a liquid-metal/molten salt extraction process with 
relatively low removal rates (cycle time $> 50$ days). This is accomplished 
by directing a small fraction of the salt mass flow leaving the nickel mesh 
filter (10\%) to the liquid-metal/molten salt component of the reprocessing 
system, where lanthanides are removed with a specific extraction efficiency to 
match the  required cycle time (Table~\ref{tab:reprocessing_list}). The rest 
90\% of the salt mass flow is directed from the nickel filter to heat 
exchanger without performing any fuel salt treatment.

The removal rates vary among nuclides in this reactor concept, which dictate 
the necessary resolution of depletion calculations. To compromise, a 3-day 
time was selected for the long-term demonstration case based on a timestep 
refinement study by Betzler \emph{et al.} \cite{betzler_assessment_2017-1}.

\section{Long-term depletion demonstration and validation}
\subsection{Constant, ideal extraction efficiency case}
To validate SaltProc v1.0, I performed lifetime-long depletion calculation 
with ideal extraction efficiency. This case was selected to repeat fuel salt 
depletion as close as possible to ChemTriton simulation for full-core 
\gls{TAP} core by Betzler \emph{et al.} \cite{betzler_assessment_2017-1}.  
Betzler \emph{et al.} made following assumptions and approximations in their 
work \cite{betzler_assessment_2017-1}:
\begin{enumerate}[noitemsep]	
	\item Effective cycle times as prescribed by Transatomic Power Technical 
	White Paper (Table~\ref{tab:reprocessing_list}) with \textbf{100\% removal 
	efficiency} for all removal groups, including noble gases.
	\item 5\% \gls{LEU} feed rate is equal to rate of fisson product removal.
	\item 3-day depletion step.
	\item Quarter-core model with vacuum boundary condition.
	\item Delayed neutron precursor drift was neglected.
\end{enumerate}
I adopted those assumption for code-to-code verification of SaltProc v1.0 
against ChemTriton. ENDF/B-VII.1 \cite{chadwick_endf/b-vii.1_2011} nuclear 
data library is used for this case to be consistent with Betzler's work.
Unfortunately, some crucial details has not been reported in 
\cite{betzler_assessment_2017-1}: exact core geometries for various moderator 
rod configurations except startup configuration, excess reactivity at startup, 
S($\alpha, \beta$) data and temperature used. This section presented my best 
effort to repeat Betzler \emph{et al.} simulation using same input data to 
validate SaltProc for the \gls{TAP} concept.


\subsubsection{Effective multiplication factor dynamics}
Figures~\ref{fig:keff-ben-valid} and \ref{fig:keff-ben-valid-zoomed} 
demonstrate the effective multiplication factor obtained using SaltProc v1.0 
with Serpent. The $k_{eff}$ was obtained after removing fission products and 
adding feed material at the end of each depletion step (3 days for this case). 
SaltProc v1.0 updated the moderator rod configuration to the next 
configuration (e.g., from 1388 rods per core to 1624 rods per core) once 
predicted value of $k_{eff}$ at the end of next depletion step drops below 1. 
This algorithm mimics regular maintenance shutdown when the \gls{TAP} core 
excess reactivity is exhausted, and moderator rod assemblies should be 
reconfigured to operate next cycle. 

Optimal number of moderator configurations (cycles) is found to be 15 (see 
Appendix~\ref{appex:geometries}). Fewer cycles would improve capacity factor 
but needs larger excess reactivity at the \gls{BOC} which is strictly limited 
by reactivity control system worth. More cycles would require frequent 
moderator rods reconfiguration which worsen capacity factor.  The interval 
between first and second moderator configuration was only 12 months, the 
shortest interval between moderator configuration updates. For the operation 
interval between 2 and 16 years after startup, the intervals between shutdowns 
for moderator rod updates was 18-26 months. But towards the \gls{EOL}, the 
intervals between moderator rod reconfiguration dropped to 13 months. Overall, 
average interval between regular shutdowns for the core reconfiguration was 18 
months which exactly matches refueling interval for conventional \glspl{LWR}  
and consistent with Betzler \emph{et al.} ($\approx$16 months)  
\cite{betzler_assessment_2017-1}.
\begin{figure}[htp!] % replace 't' with 'b' to 
	\centering
	\includegraphics[width=\textwidth]{ch4/keff_ben.png}
	\caption{Effective multiplication factor dynamics for full-core \gls{TAP} 
		core model for the case with ideal removal efficiency of fission 
		product. 
		Confidence interval $\sigma=28$ $pcm$ is shaded.}
	\label{fig:keff-ben-valid}
\end{figure}

The $k_{eff}$ fluctuates significantly as a result of the batch-wise nature of 
the online reprocessing approach used. Loading the initial fuel salt 
composition with 5\% \gls{LEU} into the \gls{TAP} core leads to a 
supercritical configuration with an excess reactivity of about 3200pcm 
(Figure~\ref{fig:keff-ben-valid}). Without performing any fuel salt 
reprocessing and spectrum shifting, the core became subcritical after 30 days 
of operation \cite{rykhlevskii_milestone_2019}. SaltProc calculates an 
operational lifetime of 22.5 years, after which the fuel salt reached a total 
burnup of 81.46 MWd/kgU. The end of operational lifetime is achieved when the 
minimum \gls{SVF} is obtained, as restricted by the moderator geometry 
parameters (e.g., moderator rod diameter, rod pitch, internal diameter of the 
reactor vessel). Table~\ref{tab:valid_ben_lifetime} compares obtained 
results with Betzler \emph{et al.} \cite{betzler_assessment_2017-1}. Overall, 
SaltProc-calculated operational lifetime and burnup are lower than the 
reference by approximately 22\% and 17\%, respectively. Better match in the 
operational lifetime between SaltProc v1.0 and ChemTriton can be obtained if 
detailed moderator configuration description of Betzler \emph{et al.} model 
will be available in a future.
\begin{figure}[htp!] % replace 't' with 'b' to 
	\centering
	\includegraphics[width=0.9\textwidth]{ch4/keff_ben_zoomed.png}
	\caption{Zoomed effective multiplication factor for the interval from 660 
		to 715 EFPD during transitioning from Cycle \#1 (startup geometry 
		configuration, 347 moderator rods, \gls{SVF}=0.91720353) to Cycle \#2 
		(\gls{SVF}=0.88694). Confidence interval $\sigma=28$ $pcm$ is 
		shaded.}
	\label{fig:keff-ben-valid-zoomed}
\end{figure}
%%%%%%%%%%%%%%%%%%%%%%%%%%%%%%%%%%%%%%%%
\begin{table}[hbp!]
	\centering
	\caption{Comparison of main operational parameters in the \gls{TAP} 
	reactor between the current work and Betzler \emph{et al.}
	\cite{betzler_assessment_2017-1}.}
	\begin{tabularx}{\textwidth}{p{0.42\textwidth} R R}
		\hline
		\textbf{Parameter}  & \textbf{Current work} & \textbf{Betzler, 2017} 
		\cite{betzler_assessment_2017-1}\\ \hline
		Operational lifetime [y] & 22.5 & 29.0 \\
		Discharge burnup [MWd/kgU] & 76.30& 91.9 \\
		Average interval between moderator reconfiguration [months] & 18 & 
		16 \\
		\hline
	\end{tabularx}
	\label{tab:valid_ben_lifetime}
	\vspace{-0.9em}
\end{table}
%%%%%%%%%%%%%%%%%%%%%%%%%%%%%%%%%%%%%%%%%%%%%%%%%%%%%%%%%%%%%%%%%%%%%%%%%%%%%%%


\subsubsection{Fuel salt isotopic composition dynamics}
Figures~\ref{fig:u-ben-valid}, \ref{fig:pu-ben-valid}, and 
\ref{fig:pu-fiss-ben-valid} show that continuous \gls{LEU} feed into the 
\gls{TAP} 
reactor is not sufficient to maintain the fissile material content of the 
core, as the uranium enrichment steadily decreases from 5\% at the \gls{BOL} 
to 1\% at the \gls{EOL}. However, during the first 13 years of operation, 
the \gls{TAP} \gls{MSR} breeds fissile $^{239}$Pu and $^{241}$Pu, reaching a 
peak of total fissile plutonium inventory of 2.15t  
(Figure~\ref{fig:pu-fiss-ben-valid}). A significant amount of non-fissile 
plutonium ($^{238}$Pu, $^{240}$Pu, and $^{242}$Pu) and uranium ($^{236}$U) 
builds up in the reactor during operation and negatively impacts criticality 
of the reactor. $^{239}$Pu and $^{241}$Pu are major contributors to fissile 
material content of the core keeping it critical during second half of 
the operational lifecycle. The total $^{239}$Pu inventory in the core rises 
during the first 11 years of operation due to the harder neutron spectrum. 
After 11 years, the softer spectrum breeds less $^{239}$Pu from $^{238}$U, and 
more of $^{239}$Pu is progressively burned. Obtained results are in a good 
agreement with results in ORNL Report by Betzler \emph{et al.} 
(Table~\ref{tab:valid_ben_isos}) \cite{betzler_assessment_2017-1}.

%$^{235}$U inventory in Betzler \emph{et al.} changed from 6.8t at the 
%\gls{BOL} to 1.0t at the \gls{EOL}. $^{239}$Pu  was 1.065t at the 
%\gls{EOL}. $^{240}$Pu  was 995kg at the 
%\gls{EOL}. $^{241}$Pu  was 465kg at the 
%\gls{EOL}.		
\begin{figure}[hbp!] % replace 't' with 'b' to 
	\centering
	\includegraphics[width=\textwidth]{ch4/u_ben_valid.png}
	\caption{SaltProc-calculated uranium isotopic fuel salt content during 
	22.5 years of operation.}
	\label{fig:u-ben-valid}
\end{figure}
\begin{figure}[hbp!] % replace 't' with 'b' to 
	\centering
	\includegraphics[width=\textwidth]{ch4/pu_ben_valid.png}
	\caption{SaltProc-calculated plutonium isotopic fuel salt content during 
		22.5 years of operation.}
	\label{fig:pu-ben-valid}
\end{figure}

\begin{figure}[hbp!] % replace 't' with 'b' to 
	\centering
	\includegraphics[width=\textwidth]{ch4/tot_pu_ben_valid.png}
	\caption{SaltProc-calculated fissile and non-fissile plutonium fuel salt 
	content during 22.5 years of operation.}
	\label{fig:pu-fiss-ben-valid}
\end{figure}
%%%%%%%%%%%%%%%%%%%%%%%%%%%%%%%%%%%%%%%%
\begin{table}[hbp!]
	\centering
	\caption{Comparison of major heavy isotopes inventories at the \gls{EOL} 
	in the \gls{TAP} reactor between the current work and Betzler \emph{et al.}	
	\cite{betzler_assessment_2017-1}.}
	\begin{tabularx}{\textwidth}{L p{0.12\textwidth} R R R}
		\hline
		& \textbf{Isotope}  & \textbf{Current work [kg]} & \textbf{Betzler, 
		2017 [kg]} & \textbf{$\Delta m$ [\%]}\\ \hline
		\multirow{4}{*}{Fissile}
		&$^{235}$U  & 1299 & 1160 & $+11$\% \\
		&$^{239}$Pu & 942  & 995  & $-5$\% \\
		&$^{241}$Pu & 427  & 435  & $-2$\% \\
		&Total & 2668 & 2590 & $+3$\%  \\ \hline
		
		\multirow{4}{*}{Non-fissile}
		&$^{236}$U  & 1123 & 1200 & $-6$\% \\
		&$^{238}$U  & 127'353 & 132'400 & $-4$\% \\
		&$^{238}$Pu & 235  & 280  & $-16$\% \\
		&$^{240}$Pu & 503  & 1000  & $-50$\% \\
		&$^{242}$Pu & 230  & 310  & $-26$\% \\
		&Total & 129'444 & 135'190 & $+4$\%  \\ \hline
	\end{tabularx}
	\label{tab:valid_ben_isos}
	\vspace{-0.9em}
\end{table}
%%%%%%%%%%%%%%%%%%%%%%%%%%%%%%%%%%%%%%%%%%%%%%%%%%%%%%%%%%%%%%%%%%%%%%%%%%%%%%%

Lifetime-long SaltProc calculation requires a 5\% \gls{LEU} feed rate of 460.8 
kg per year to maintain the fuel salt inventory in the primary loop which is 
consistent with the reference. Table~\ref{tab:valid_ben_performance} shows 
main fuel cycle performance parameters calculated using SaltProc and compared 
with the reference. Normalized per GW$_{th}$-year, the \gls{TAP} concept 
requires about 5.23t of fuel compared with 4.14t reported by Betzler \emph{et 
al.} SaltProc-calculated waste production normalized per GW$_{th}$-year is 
5\% less than reported by ORNL. Potentially, the \gls{TAP} can operate with 
\gls{LWR} \gls{SNF} as the fissile material feed. The heavy metal component of 
\gls{LWR} \gls{SNF} has a lower fissile material weight fraction than 5\% 
enriched uranium and adds less fertile $^{238}$U to the fuel salt, potentially 
reducing the operational lifetime. But in case of using waste material (e.g., 
Transuranium elements from \gls{LWR} \gls{SNF}) in this fueling scenario, the 
\gls{TAP} concept has the potential to have a better waste reduction metrics.
%%%%%%%%%%%%%%%%%%%%%%%%%%%%%%%%%%%%%%%%
\begin{table}[hbp!]
	\centering
	\caption{Comparison of normalized total fuel load and actinide waste from 
	the TAP reactor obtained in the current work and Betzler \emph{et al.} 
	\cite{betzler_assessment_2017-1}.}
	\begin{tabularx}{\textwidth}{p{0.42\textwidth} R R}
		\hline
		\textbf{Parameter}  & \textbf{Current work} & \textbf{Betzler, 2017} 
		\cite{betzler_assessment_2017-1}\\ \hline
		5\% \gls{LEU} feed rate [kg/y] & 460.8 & 480.0 \\
		Loaded fuel [MT per GW$_{th}$-y] & 5.23 & 4.14 \\
		Waste  [MT per GW$_{th}$-y] & 3.57 & 3.74 \\
		\hline
	\end{tabularx}
	\label{tab:valid_ben_performance}
	\vspace{-0.9em}
\end{table}
%%%%%%%%%%%%%%%%%%%%%%%%%%%%%%%%%%%%%%%%%%%%%%%%%%%%%%%%%%%%%%%%%%%%%%%%%%%%%%%

\subsubsection{Neutron energy spectrum}
Significant thermalization of the neutron spectrum is observed as moderator 
rods are added into the core configuration 
(Figure~\ref{fig:ben-spectrum-bol}). At startup, the neutron spectra from the 
current work and Betzler \emph{et al.} are match well because the core 
geometry, its \gls{SVF}, and initial fuel composition in 
these two simulation are similar. Pearson correlation  
coefficient\footnote{Pearson correlation coefficient is calculated by the 
	following formula:
	\begin{align}
	r &= \frac{\sum_{i=1}^{N} 
		(\Phi_i^{ref}-\overline{\Phi^{ref}})(\Phi_i-\overline{\Phi})}
	{\sqrt{\sum_{i=1}^{N} (\Phi_i^{ref}-\overline{\Phi^{ref}})^2 
			\sum_{i=1}^{N} 
			(\Phi_i-\overline{\Phi})^2}}\\
	\mbox{where} \nonumber\\
	\Phi_i^{ref},\Phi_i &= \mbox{neutron flux for i$^{th}$ energy bin 
		reported in the reference and the current work $[n/cm^2\cdot s]$} 
	\nonumber\\
	\overline{\Phi^{ref}}, \overline{\Phi} &= \mbox{neutron flux averaged over 
		N energy bins reported in the reference and current work $[n/cm^2\cdot 
		s]$} 
	\nonumber\\
	N &= \mbox{number of neutron energy bins [-].}
	\nonumber
	\end{align}}
$r_{BOL}=0.91115$ which indicates strong, positive association between the 
spectra at the \gls{BOL} (see Figure~\ref{fig:ben-spectrum-bol}, upper plot).
At the \gls{EOL}, SaltProc/Serpent-calculated spectrum is more thermal than 
reported by Betzler \emph{et al.} \cite{betzler_assessment_2017-1}, but 
correlation coefficient $r_{EOL}=0.90987$ shows that the spectra are still 
extremely strongly related (see Figure~\ref{fig:ben-spectrum-bol}, lower 
plot). 
%with smaller amplitude of resonances between 10$^{-5}$ and 10$^{-2}$ MeV 
%(resonance capture of neutrons by $^{238}$U).
\begin{figure}[htbp!] % replace 't' with 'b' to 
	\centering
	\includegraphics[width=0.77\textwidth]{ch4/ben_spec_bol.png}\\
	\vspace{-12mm}
	\hspace{0.5mm}
	\includegraphics[width=0.77\textwidth]{ch4/ben_spec_eol.png}
	\vspace{-3mm}
	\caption{Neutron flux energy spectrum at the BOL (upper) and the EOL 
		(lower) obtained using SaltProc/Serpent (orange) compared with 
		ChemTriton/Shift (blue) \cite{betzler_assessment_2017-1}.}
	\label{fig:ben-spectrum-bol}
\end{figure}


Faster spectrum at the \gls{BOL} tends to significantly increase resonance 
absorption
in $^{238}$U and decrease the absorptions in fissile and 
construction materials. Thus, softer spectrum in the current work compared 
with Betzler \emph{et al.} led to fewer resonance captures\footnote{Energy 
range for $^{238}$U resonance neutron capture is between 10$^{-5}$ and 
10$^{-2}$ MeV.} of neutrons by $^{238}$U, hence, less $^{239}$Pu bred from 
$^{238}$U. Therefore, the SaltProc/Serpent calculation in the current work 
underpredicts the destruction (i.e., fission and capture) of $^{235}$U and 
overpredicts the destruction of $^{235}$U (see Table~\ref{tab:valid_ben_isos}). 
Finally, the softer neutron spectrum leds to more fissions in fissile 
plutonium isotopes ($^{239}$Pu and $^{241}$Pu) which also decreases 
non-fissile plutonium (Table~\ref{tab:valid_ben_isos}) and total actinide 
waste production (Table~\ref{tab:valid_ben_performance}).


\subsubsection{Time step refinement}
The results shown in this chapter are obtained from a SaltProc calculation 
with uniform depletion time step of 3 days. The duration of the time step was 
chosen after performing parametric sweep to determine the longest depletion 
time step that provides the accuracy of the calculation. Larger time step 
potentially reduces the SaltProc calculation costs, providing results faster 
for lifetime-long (25-year) simulations. 

Figure~\ref{fig:timeref-keff} shows $k_{eff}$ evolution obtained with 3-, 6-, 
12-, and 24-days depletion time intervals for 25-year simulation. The interval 
between moderator configuration updates was assumed similar for all four cases 
for consistency. Multiplication factor at the \gls{BOC} for each moderator 
configuration became lower when the time step increasing. At the \gls{EOC} for 
each geometry, $k_{eff}=1.0$  for 3-day time step and drops significantly below 
1.0 when depletion interval rises. The use of longer depletion time step 
decreases $k_{eff}$ at the end of each depletion step. The decrease is because 
more poisonous \glspl{FP} (e.g., $^{135}$Xe) are produced in the core. With 
longer time steps, large concentration of poisons are obtained at the end of 
depletion step when those poisons are being removed, resulting in the 
criticality growth. 

Figures~\ref{fig:timeref-u} and \ref{fig:timeref-u} show that while the longer 
time steps appropriately capture uranium depletion ($<1$\% difference even for 
24-day time step), observed difference in fissile $^{239}$Pu mass is 
significant for the depletion interval 6-days and longer ($>0.5$\% difference). 
Using the 6-day depletion interval leads to overprediction of $^{239}$Pu 
production by 5 kg at the \gls{EOL} (Figure~\ref{fig:timeref-pu239}). The use 
of 6-day time step tends to cause an overprediction of total plutonium 
production by 9.6 kg. Notably, significant quantity (SQ) for plutonium 
currently in use by the IAEA is 8 kg ($<80$\% $^{238}$Pu) 
\cite{close_iaea_1995}. Thus, depletion interval longer than 6 days leads to 
significant error in the predicted plutonium inventory at the \gls{EOL}, which 
is larger than SQ. 
\begin{figure}[hbp!] % replace 't' with 'b' to 
	\centering
	\includegraphics[width=\textwidth]{ch4/keff_time_refinement.png}
	\caption{SaltProc-calculated effective multiplication factor ($k_{eff}$) 
	during operation for different depletion time step sizes.}
	\label{fig:timeref-keff}
\end{figure}

\begin{figure}[hbp!] % replace 't' with 'b' to 
	\centering
	\includegraphics[width=0.91\textwidth]{ch4/u235_time_refinement.png}\\
		\vspace{-11mm}
	\hspace{0.5mm}
	\includegraphics[width=\textwidth]{ch4/u238_time_refinement.png}
		\vspace{-6mm}
	\caption{SaltProc-calculated $^{235}$U (upper) and $^{238}$U (lower) 
	content during operation for different depletion time step sizes.}
	\label{fig:timeref-u}
\end{figure}

\begin{figure}[htp!] % replace 't' with 'b' to 
	\centering
	\includegraphics[width=\textwidth]{ch4/pu239_time_refinement.png}
	\caption{SaltProc-calculated $^{239}$Pu content during operation for 
		different depletion time step sizes.}
	\label{fig:timeref-pu239}
\end{figure}
Increasing depletion time interval significantly reduces computational cost but 
also significantly deteriorates the accuracy of depletion 
calculations. Calculations using a depletion time step of 6 days and longer 
demonstrated a significant difference in calculated $k_{eff}$ and depleted mass 
from those using 3-day depletion step. In the current work, the 3-day 
depletion step was chosen to adequately predict the mass of major heavy 
isotopes in the fuel salt during 25 years of the \gls{TAP} reactor operation.


\subsection{Realistic extraction efficiency case}

\subsubsection{Effective multiplication factor}
Figure~\ref{fig:keff-eps-var} and \ref{fig:keff-eps-var-zoom} show $k_{eff}$ 
dynamics for the \gls{TAP} core with realistic removal efficiency.

\begin{figure}[htp!] % replace 't' with 'b' to 
	\centering
	\includegraphics[width=\textwidth]{ch4/eps/keff.png}
	\caption{Effective multiplication factor dynamics for full-core \gls{TAP} 
		core model for the case with realistic removal efficiency of fission 
		product and various mass transfer coefficients ($K_L$). 
		Confidence interval $\sigma=28$ $pcm$ is shaded.}
	\label{fig:keff-eps-var}
\end{figure}

\begin{figure}[hbp!] % replace 't' with 'b' to 
	\centering
	\includegraphics[width=\textwidth]{ch4/eps/keff_zoomed.png}
	\caption{Zoomed effective multiplication factor dynamics during switching 
	from Cycle \#14 (\gls{SVF}=0.582846) to  Cycle \#15 (\gls{SVF}=0.535505) 
	for the case with realistic removal efficiency of fission 
		product and various mass transfer coefficients ($K_L$). 
		Confidence interval $\sigma=28$ $pcm$ is shaded.}
	\label{fig:keff-eps-var-zoom}
\end{figure}

\subsubsection{Fuel salt isotopic composition}

\section{Reactor load following analysis}

\section{Prototype design for the xenon removal system}

\section{Concluding remarks}