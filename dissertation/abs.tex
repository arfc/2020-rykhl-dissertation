Nuclear reactors with liquid fuel offer multiple advantages over their 
solid-fueled siblings: improved inherent safety, fuel utilization, thermal
efficiency, online reprocessing, and potential for nuclear fuel cycle closure. 
To advance this promising reactor design, researchers need a simulation tool 
for fuel depletion calculations while taking into account online reprocessing 
and refueling.

This work presents a flexible, open-source tool, SaltProc, for simulating the 
fuel depletion in a generic nuclear reactor with liquid, circulating fuel.
SaltProc allows the user to define realistically constrained extraction 
efficiency of fission products based on physical models of fuel processing 
components appearing in various \gls{MSR} systems.  Developed using a Python 
Object-Oriented Programming paradigm, SaltProc can model a complex, 
multi-zone, multi-fluid \gls{MSR} operation and is sufficiently general to 
represent myriad reactor systems. Moreover, SaltProc can maintain reactor 
criticality by adjusting the geometry of the core.  Finally, the tool can 
analyze power variations in the context of depletion.

This thesis also demonstrates and validates SaltProc for two prospective 
reactor designs: the \gls{MSBR} and the \gls{TAP} \gls{MSR}. A 60-year 
full-power \gls{MSBR} depletion calculation with ideal fission product 
extraction (e.g., 100\% of target poison removed) has been validated against 
Betzler \emph{et al.} simulation results obtained with ChemTRITON at ORNL. The 
average $^{232}$Th feed rate obtained is the current work is 2.40 kg/d, which 
is consistent with ORNL results (2.45 kg/d). This simulation showed that the 
online fission product extraction and online refueling with $^{232}$Th allowed 
the \gls{MSBR} to operate at full power for 60 years due to exceptionally low 
parasitic neutron absorption.

This work shows fuel depletion simulations with SaltProc for the \gls{TAP} 
\gls{MSR} to demonstrate the tool capability to model liquid-fueled reactors 
with movable/adjustable moderator. This dissertation also validated depletion 
calculations for a realistic multi-component model of the fuel salt 
reprocessing system with assumed ideal extraction efficiency against full-core 
\gls{TAP} depletion analysis by Betzler \emph{et al.} from ORNL. The average 
SaltProc-calculated 5\%-enriched uranium feed rate is 460.8 kg/y, which agrees 
well with the reference (480 kg/y). 

This dissertation illuminated the impact of xenon extraction efficiency on the 
long-term fuel cycle performance for the realistic reprocessing system model 
of the TAP concept with non-ideal removal efficiency. For limited gas removal 
efficiency, the fuel salt composition is strongly influenced by the neutron 
spectrum hardening due to the presence of neutron poisons ($^{135}$Xe) in the 
core. Thus, more effective noble gas extraction significantly reduced neutron 
loss due to parasitic absorption, which led to better fuel utilization and 
extended core lifetime. 

Additionally, this work investigated \gls{MSR} load-following capability 
through short-term depletion analysis with the power level variation 
$P\in[0,100\%]$. Online gas removal significantly improved the load-following 
capability of the \gls{MSBR} by reducing xenon poisoning from $-1457$ $pcm$ to 
$-189$ $pcm$. The \gls{TAP} \gls{MSR} demonstrated a negligible 
xenon poisoning effect even without online gas removal because its neutron 
energy spectrum is relatively fast throughout its lifetime.

This work also analyzed safety parameter (temperature and void coefficient of 
reactivity, total control rod worth, kinetic parameters) variation during 
operation using fuel composition evolution obtained with SaltProc. On a 
lifetime-long timescale, the safety parameters worsened during operation for 
both considered \glspl{MSR} due to a significant spectral shift. On a 
short-term timescale, the safety parameters during \gls{MSBR} 
load-following slightly worsened right after power drop because $^{135}$Xe 
concentration peak caused substantial neutron spectrum hardening. However, 
during the next few hours, the gas removal system removed almost all 
$^{135}$Xe from the fuel, which led to significant improvement in all safety 
parameters. Overall, a reduced amount of neutron poisons (e.g., $^{135}$Xe) 
due to online gas extraction improved the safety case for both \gls{MSR} 
designs.

Finally, a simple uncertainty propagation via Monte Carlo depletion 
calculations in this work showed that the nuclear-data-related error (0.5-8\% 
depending on the nuclide) is two orders of magnitude greater than the 
stochastic error ($<0.07$\%).