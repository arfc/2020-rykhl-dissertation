Nuclear reactors with liquid fuel offer multiple advantages over their 
solid-fueled siblings: improved inherent safety, fuel utilization, thermal
efficiency, online reprocessing, and potential for nuclear fuel cycle closure. 
Interest in advanced liquid-fueled nuclear systems, particularly \glspl{MSR}, 
has grown recently with numerous new commercial \gls{MSR} concepts. However, 
most contemporary reactor physics software is incapable of performing fuel 
depletion calculations for such advanced reactors. To further develop 
liquid-fueled reactor designs, researchers need a simulation tool for 
performing fuel depletion calculations while taking into account online fuel 
reprocessing and refueling. 

This work presents a flexible, open-source tool, SaltProc, for simulating the 
fuel depletion in a generic nuclear reactor with liquid, circulating fuel.
SaltProc allows the user to define realistically constrained extraction 
efficiency of fission products based on physical models of fuel processing 
components appearing in various \gls{MSR} systems.  Developed using Python 
Object-Oriented Programming paradigm, SaltProc can model a complex, 
multi-zone, multi-fluid \gls{MSR} operation and is sufficiently general to 
represent myriad of reactor systems. Moreover, SaltProc can maintain reactor 
criticality by adjusting the geometry of the core.  Finally, the tool can 
perform fuel depletion calculations during power changes.

SaltProc is demonstrated and validated for two perspective reactor designs: 
\gls{MSBR} and \gls{TAP} \gls{MSR}. A 60-year full-power \gls{MSBR} depletion 
calculation with ideal fission product extraction (e.g., 
100\% of target poison is being removed) has been validated against Betzler 
\emph{et al.} simulation results obtained with ChemTRITON at ORNL. The average 
$^{232}$Th feed rate obtained in the current work is about 2.40 kg/d, which is 
consistent with ORNL results (2.45 kg/d). This simulation showed that the 
online fission product extraction and online refueling with $^{232}$Th allowed 
the \gls{MSBR} to operate at full power for 60 years due to exceptionally low 
parasitic neutron absorption.

Fuel depletion simulation with SaltProc for the \gls{TAP} \gls{MSR} have 
demonstrated the tool capability to model liquid-fueled reactors with 
movable/adjustable moderator. The results for a realistic multi-component 
model of the fuel salt reprocessing system with ideal extraction 
efficiency are validated against full-core \gls{TAP} depletion analysis by 
Betzler \emph{et al.} from ORNL. The average SaltProc-calculated 5\%-enriched 
uranium feed rate is 460.8 kg/y, which agrees well with the reference (480 
kg/y). %Finally, the discrepancies in calculated isotopic composition after 25 
%years of full-power operation between the current work and the reference are 
%is 3\% and 4\% for fissile and non-fissile nuclides, respectively.

In this dissertation, the realistic reprocessing system with non-ideal 
removal efficiency was modeled for the TAP reactor to illuminate the impact of 
xenon extraction efficiency on the long-term fuel cycle performance. For the 
low gas removal efficiency, the fuel salt composition is strongly influenced 
by the neutron spectrum hardening due to the presence of neutron poisons 
($^{135}$Xe) in the core. Thus, more effective noble gas extraction 
significantly reduced neutron loss due to parasitic absorption, which led to 
better fuel utilization and extended core lifetime. 

Additional short-term depletion analysis with the power level variation 
$P\in[0,100\%]$ was performed using SaltProc to investigate \glspl{MSR} 
load-following capability. Online gas removal 
significantly improved the load-following capability of the \gls{MSBR} by 
reducing xenon poisoning from $-1457$ $pcm$ to $-189$ $pcm$. Contrarily, the 
\gls{TAP} \gls{MSR} demonstrated a negligible xenon poisoning effect even 
without online gas removal because its neutron energy spectrum is relatively 
fast throughout a lifetime.

Obtained fuel composition evolution has been used to analyze safety parameters 
(temperature and void coefficients of reactivity, total control rod worth, 
kinetic parameters) variation during operation. On a lifetime-long timescale, 
the safety parameters worsened during operation for both considered 
\glspl{MSR} due to a significant spectral shift. On a short-term timescale, 
the safety parameters during the \gls{MSBR} load-following slightly worsened 
right after power drop because $^{135}$Xe concentration peak caused 
substantial neutron spectrum hardening. However, during the next few hours, 
the gas removal system removed almost all $^{135}$Xe from the fuel, which led 
to significant improvement of all safety parameters. Overall, a reduced amount 
of neutron poisons (e.g., $^{135}$Xe) due to online gas extraction improved 
the reactors safety.

Finally, a simple uncertainty propagation via Monte Carlo depletion 
calculations in this work showed that the nuclear data-related error (0.5-8\% 
depending on the nuclide) is two orders of magnitude greater than the 
stochastic error ($<0.07$\%).