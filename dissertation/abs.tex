Nuclear reactors with liquid fuel offering multiple advantages over their 
solid-fueled siblings: an improved inherent safety, fuel utilization, thermal 
efficiency, online reprocessing, and potential for nuclear fuel cycle closure. 
The interest in advanced liquid-fueled nuclear 
systems, particularly \glspl{MSR}, has grown recently with numerous new 
commercial \gls{MSR} concepts. 
However, most contemporary reactor physics software is incapable of performing 
fuel depletion calculations for such advanced reactors. To further 
develop liquid-fueled reactor designs, researchers need a simulation tool for 
performing fuel depletion calculations while taking into account online fuel 
reprocessing and refueling. Current modeling efforts in the literature usually 
assume ideal
(e.g., 100\% of neutron poison being removed) rather than 
realistically constrained removal
efficiency. 

This work aims to provide a flexible tool, SaltProc, for simulating 
the fuel depletion in a generic nuclear reactor with liquid, circulating fuel. 
SaltProc allows the user to specify realistically constrained extraction 
efficiency of fission products based on physical models of
fuel processing 
components appearing in various \gls{MSR} systems. Moreover, SaltProc can 
maintain the reactor's criticality by adjusting the geometry of the core. This 
work demonstrated and validated SaltProc for lifetime-long and short-term 
depletion calculation in two \gls{MSR} designs: \gls{MSBR} and \gls{TAP} 
\gls{MSR}. SaltProc successfully captured the evolution of fuel salt 
composition and major safety parameters (temperature and void coefficient of 
reactivity, shutdown margin) during reactor operation with various efficiency 
of noble gas extraction.