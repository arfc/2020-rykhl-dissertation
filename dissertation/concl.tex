\chapter{Conclusions and future work}

\section{General Conclusions}
Liquid-fueled nuclear reactors offer a number of advantages over their 
traditional solid-fueled counterparts, which makes them a promising option for 
nuclear fuel cycle closure while offering improved inherent safety.

This work demonstrated a flexible, open-source tool, SaltProc, capable of 
simulating fuel depletion in a wide range of circulating-fuel (e.g., liquid 
fuel which circulates throughout the primary loop) nuclear reactors with 
taking into account unique feature of such systems: online fuel reprocessing 
and refueling. SaltProc extends the continuous-energy Monte Carlo Burnup 
calculation code, Serpent 2, for the simulation of material
isotopic evolution 
in any nuclear reactors with circulating, liquid fuel with the main focus on 
the liquid-fueled \glspl{MSR}. This work demonstrates a clear contribution to 
the nuclear engineering community by providing a tool for fuel depletion 
calculations in any generic nuclear system with circulating liquid fuel.

The need for this work has been shown by a summary of the current state of the 
art of \gls{MSR} depletion simulator capabilities. The literature review in 
Chapter 1 concluded that most \gls{MSR} depletion simulators typically assume 
ideal (rather than realistically constrained) poison removal rates for the 
nuclear system performance modeling. Moreover, most of the simulators assumed 
constant extraction efficiency vectors, which must be determined by the user 
in the input file and cannot be a function of other parameters. SaltProc is 
capable to model the peculiarities of \gls{MSR}, namely:
complex, multi-component reprocessing system structure and realistically 
constrained extraction efficiency of fission product described as function of 
many parameters. Furthermore, SaltProc has capability to maintain the reactor 
criticality by adjusting the reactor core geometry. In addition to fundamental 
simulation capabilities, SaltProc has scalable design and allows development 
of additional advanced capabilities in the future.

I demonstrated SaltProc for lifetime-long full-power operation 
for two perspective \gls{MSR} designs: \gls{MSBR} and \gls{TAP} \gls{MSR}. The 
\gls{MSBR} analysis illuminated simplified depletion of the fuel salt for 60 
years of full-power operation with ideal fission product extraction efficiency
(e.g., 100\% of target poison is being removed). The online fission product 
removal with 100\% efficiency and fresh fuel feed allowed the \gls{MSBR} 
operate on full-power extremely long time with effective fuel utilization due 
to exceptionally low parasitic neutron absorption. The obtained results are 
validated with published modeling efforts by \gls{ORNL} 
\cite{betzler_molten_2017}.

Validation simulations for the \gls{TAP} \gls{MSR} have demonstrated SaltProc 
capability to model reactors with adjustable moderator configuration. Results 
for realistic multi-component model of the fuel salt reprocessing system with 
assumed ideal removal efficiency are validated with full-core \gls{TAP} 
depletion analysis by Betzler \emph{et al.} \cite{betzler_assessment_2017-1}. 
In the realistic reprocessing system with non-ideal removal, the fuel salt 
composition is strongly influenced by the neutron spectrum hardening due to 
presence of neutron poisons (e.g., $^{135}$Xe) in the core. Thus, more 
effective noble gas extraction efficiency significantly reduce neutron loss 
due to parasitic absorption which led to
better fuel utilization and extended 
core lifetime.

Additionally, the short-term depletion analysis with power maneuvering in
[0,100\%] range is performed with SaltProc to investigate load-following 
capability in the \gls{TAP} \gls{MSR} and \gls{MSBR} designs. Online gaseous 
fission products removal significantly improved the load-following capability 
of the \gls{MSBR} by reducing the reactivity worth of xenon poisoning from 
$-1457$ $pcm$ to $-189$ $pcm$. In the \gls{TAP} \gls{MSR}, I observed 
negligible effect of xenon poisoning because neutron energy spectrum in the 
\gls{TAP} core is relatively hard even for the most thermal core configuration 
(all moderator rods are inserted). Thus, the \gls{TAP} \gls{MSR} can 
effectively load-follow even without continuous gas removal.

Once fuel salt composition evolution is obtained for various \gls{MSR} 
designs and power levels, I analyzed a major safety and operational parameters 
at different moments during operation. Specifically, changes in temperature 
and void coefficients of reactivity and total control rod worth are evaluated 
for the \gls{TAP} concept and \gls{MSBR} for two time frames: lifetime-long 
full-power operation and short-term load-following transient. In 
long-time-scale, the safety parameters worsened during full-time operation for 
both considered reactor designs due to significant spectral shift. For the 
load-following transient, the combination of fuel and moderator temperature 
coefficient is remained strongly negative throughout the transient for both 
reactors. Notably, the \gls{MSBR} safety benefited from continuous fission gas 
removal, while the \gls{TAP} \gls{MSR} safety and operational parameters 
remained stable due to its harder spectrum. Unfortunately, the total control 
rod worth is insufficient to shut down the \gls{MSBR} due to huge reactivity 
swing during load-following transient. Thus, the reactivity control system of 
the \gls{MSBR} must be redesigned to ensure safe power maneuvering.

The current work also demonstrated a simple uncertainty propagation via Monte 
Carlo depletion calculations. I separately evaluated uncertainties on 
predicted isotopic composition from two main sources: stochastic error from 
the transport problem solution and measurement error in the nuclear data 
library. Nuclear data-related uncertainty in the isotopic masses is 
approximately 0.5-8\% varies largely from isotope to isotope due to wide 
spread in the nuclear data covariances. The stochastic errors in isotopic 
masses are below 0.07\% for a reasonable number of neutron histories 
($7.5\times 10^6$). Fundamentally, we do not need to waste
a huge 
computational power to simulate large number of neutron histories per each 
depletion step
because the nuclear data-related uncertainty is dominating over 
the stochastic error.




\section{Suggested Future Work}
Continued research into SaltProc-Serpent and related topics could progress
in a number of different directions. First of all, another liquid-fueled 
\gls{MSR} designs with on-site fuel salt reprocessing system should be modeled 
using SaltProc to improve cross-code validation portfolio. For example, 
SaltProc can be validated with recently published effort for Chinese 
Single-fluid Double-zone Thorium Molten Salt Reactor (SD-TMSR) 
\cite{ASHRAF2019107115}.

Next, optimization of reprocessing
parameters (e.g. 
time step, feeding rate, protactinium removal
rate) could establish the best 
fuel utilization, breeding ratio, or
safety characteristics for various 
designs. This might be performed
with a parameter sweeping outer loop which 
would change an
input parameter by a small increment, run the simulation and 
analyze output to determine optimal configuration. Alternatively, the
existing 
RAVEN optimization framework \cite{alfonsi_raven_2016}
might be employed for 
such optimization studies.

Only simple power drop-and-restart transient with a coarse time resolution has 
been considered in this work to investigate load-following capabilities of 
liquid-fueled \glspl{MSR}. An additional analyses should include realistic 
power load profiles with 15-minute or even 5-minute time resolution. Existing 
capabilities of SaltProc allows modeling of smart gas separation regulation 
during transient by adjusting, for example, the helium bubble sizes in the 
sparger. The scientific community would benefit enormously from standardized 
depletion analysis during the load-following operation for various 
liquid-fueled reactors including exotic liquid metal fuel reactor designs.

Only the batch-wise online reprocessing approach has been treated in this 
work. However, the Serpent 2 was extended for continuous online fuel 
reprocessing simulation \cite{aufiero_extended_2013}. This extension could be 
employed for immediate removal of fission product gases (e.g., Xe, Kr) which 
have a strong negative impact on core lifetime and breeding efficiency. 
Finally, using the built-in Serpent 2 Monte Carlo code online reprocessing \& 
refueling material burnup routine would significantly speed up 
computer-intensive full-core depletion simulations.

%As it was pointed out, the uncertainties of the nuclear data and its impact 
%on 
%a major safety and kinetics need to be evaluated with taking into account 
%online reprocessing and refueling. The fuel motion has a large impact on 
%safety and kinetic parameters and should be carefully investigated. This 
%would 
%require development a multi-physics model of the \gls{MSR} with some advanced 
%multi-physics software such as Moltres \cite{lindsay_introduction_2018}.

Additional physical models for fission product extraction efficiency will 
enrich the capabilities of this tool.


