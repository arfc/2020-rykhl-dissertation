\chapter{Conclusions and future work}

\section{General Conclusions}
Liquid-fueled nuclear reactors offer several advantages over their traditional 
solid-fueled counterparts, which makes them a promising option for nuclear 
fuel cycle closure while offering improved inherent safety. Simulating such 
systems presents a challenge because existing reactor physics software for 
fuel burnup historically has been developed for traditional, solid-fueled 
reactors.

This work demonstrated a flexible, open-source tool, SaltProc, for 
simulating fuel depletion in a wide range of circulating-fuel (e.g., liquid 
fuel circulating throughout the primary loop) nuclear reactors that takes into 
account unique features of such systems: online fuel reprocessing 
and refueling. SaltProc extends the continuous-energy Monte Carlo burnup 
calculation code, Serpent 2, for the simulation of material isotopic evolution 
in any nuclear reactors with circulating, liquid fuel with the main focus on 
the liquid-fueled \glspl{MSR}. This work demonstrates a clear contribution to 
the nuclear engineering community by providing a tool for fuel depletion 
calculations in any generic nuclear system with circulating fuel.

The need for this work has been shown by a summary of the current state of the 
art of \gls{MSR} depletion simulator capabilities. The literature review in 
Chapter 1 concluded that most \gls{MSR} depletion simulators typically assume 
ideal (rather than realistically constrained) poison removal rates for the 
nuclear system performance modeling. Moreover, most of the simulators assumed 
constant extraction efficiency vectors, which must be determined by the user 
in the input file and cannot be a function of other parameters. SaltProc is 
capable of modeling the peculiarities of \glspl{MSR}, namely:
complex, multi-component reprocessing system structure and realistic 
extraction efficiency of fission product described as a function of 
many parameters. Furthermore, SaltProc can maintain reactor criticality by 
adjusting the reactor core geometry. In addition to fundamental simulation 
capabilities, SaltProc has a scalable design and allows the development of 
additional advanced capabilities in the future.

I demonstrated SaltProc for lifetime-long full-power operation for two 
perspective \gls{MSR} designs: \gls{MSBR} and \gls{TAP} \gls{MSR}. The 
\gls{MSBR} analysis illuminated the simplified depletion of the fuel salt for 
60 years of full-power operation with ideal fission product extraction 
efficiency (e.g., 100\% of target poison is being removed). The online fission 
product removal with 100\% efficiency and fresh fuel feed allowed the 
\gls{MSBR} to operate at full-power for an extremely long time with effective 
fuel utilization due to exceptionally low parasitic neutron absorption. The 
obtained results are validated with published modeling efforts by \gls{ORNL} 
\cite{betzler_molten_2017}.

Validation simulations for the \gls{TAP} \gls{MSR} have demonstrated the 
SaltProc capability to model reactors with adjustable moderator configuration. 
Results for a realistic multi-component model of the fuel salt reprocessing 
system with assumed ideal removal efficiency are validated with full-core 
\gls{TAP} depletion analysis by Betzler \emph{et al.} 
\cite{betzler_assessment_2017-1}. 
In the realistic reprocessing system with non-ideal removal, the fuel salt 
composition is strongly influenced by the neutron spectrum hardening due to 
presence of neutron poisons (e.g., $^{135}$Xe) in the core. Thus, more 
effective noble gas extraction efficiency significantly reduced neutron loss 
due to parasitic absorption, which led to
better fuel utilization and extended 
core lifetime.

I also used SaltProc to perform short-term depletion analysis 
with power maneuvering in the $P\in[0,100\%]$ range to investigate 
load-following capability in the \gls{TAP} \gls{MSR} and \gls{MSBR} designs. 
Online gaseous fission product removal significantly improved the 
load-following capability of the \gls{MSBR} by reducing the reactivity worth 
of xenon poisoning from $-1457$ $pcm$ to $-189$ $pcm$. I observed a negligible 
effect of xenon poisoning in the \gls{TAP} \gls{MSR} because its neutron 
energy spectrum is relatively hard even for the most thermal core 
configuration (all moderator rods are inserted). Thus, the \gls{TAP} \gls{MSR} 
can effectively load-follow even without continuous gas removal.

Once fuel salt composition evolution was obtained for various \gls{MSR} 
designs and power levels, I analyzed a major safety and operational parameters 
at different moments during operation. Specifically, changes in temperature 
and void coefficients of reactivity and total control rod worth were evaluated 
for the \gls{TAP} concept and \gls{MSBR} for two timeframes: lifetime-long 
full-power operation and short-term load-following transient. On a 
long-timescale, the safety parameters worsened during full-time operation for 
both considered reactor designs due to a significant spectral shift. For the 
load-following transient, the combination of fuel and moderator temperature 
coefficient remained strongly negative throughout the transient for both 
reactors. Notably, the \gls{MSBR} safety benefited from continuous fission gas 
removal, while the \gls{TAP} \gls{MSR} safety and operational parameters 
remained stable due to its harder spectrum. Unfortunately, the total control 
rod worth was insufficient to shut down the \gls{MSBR} due to a considerable 
reactivity swing during the load-following transient. Thus, the reactivity 
control system of the \gls{MSBR} must be redesigned to ensure safe power 
maneuvering.

The current work also demonstrated a simple uncertainty propagation via Monte 
Carlo depletion calculations. I evaluated the uncertainty of predicted 
isotopic composition separately from two primary sources: stochastic error 
from the transport problem solution and measurement error in the nuclear data 
library. Nuclear data-related uncertainty in the isotopic masses is 
approximately 0.5-8\% and varies widely from isotope to isotope due to 
widespread in the nuclear data covariances. The stochastic errors in isotopic 
masses are below 0.07\% for a reasonable number of neutron histories 
($7.5\times 10^6$). Fundamentally, we do not need to waste a substantial 
computational power to simulate a large number of neutron histories per each 
depletion step because the nuclear data-related uncertainty is dominating over 
the stochastic error. 

Furthermore, the nuclear data-related uncertainty in the 
depletion calculations can be significantly improved by reducing cross section 
covariance of $^{6}$Li, $^{7}$Li, and $^{19}$F, which are broadly used in the 
\glspl{MSR}. The nuclear data for those isotopes were not measured accurately 
because lithium and fluorine rarely appear in conventional \glspl{LWR} 
core. To further develop the \gls{MSR} concepts, $^{6}$Li, $^{7}$Li, and 
$^{19}$F cross sections must be thoroughly remeasured with improved 
uncertainty to reduce the nuclear-data related error in a neutronic 
calculations.



\section{Suggested Future Work}
Continued research into SaltProc-Serpent and related topics could progress
in many different directions. First of all, other liquid-fueled 
\gls{MSR} designs with on-site fuel salt reprocessing system should be modeled 
using SaltProc to improve the cross-code validation portfolio. For example, 
SaltProc can be validated with a recently published effort for the Chinese 
Single-fluid Double-zone Thorium Molten Salt Reactor (SD-TMSR) 
\cite{ASHRAF2019107115}.

Next, optimization of reprocessing parameters (e.g., time step, feeding rate, 
removal rate for various fission product groups) could establish the best fuel 
utilization, breeding ratio, or safety characteristics for various designs. 
This might be performed with a parameter sweeping outer loop, which would 
change an input parameter by a small increment, run the simulation, and 
analyze output to determine optimal configuration. Alternatively, the existing 
RAVEN optimization framework \cite{alfonsi_raven_2016} might be employed for 
such optimization studies.

Only the simple power drop-and-restart transient with a coarse time resolution 
has been considered in this work to investigate the load-following 
capabilities of liquid-fueled \glspl{MSR}. Additional analyses should include 
realistic power load profiles with 15-minute or even 5-minute time resolution. 
The existing capabilities of SaltProc allow modeling of smart gas separation 
regulation during transient by adjusting, for example, the helium bubble sizes 
in the sparger. The scientific community would benefit enormously from 
standardized depletion analysis during the load-following operation for 
various liquid-fueled reactors, including exotic liquid metal fuel reactor 
designs.

Only the batch-wise online reprocessing approach has been treated in this 
work. However, Serpent 2 was recently extended for continuous online fuel 
reprocessing simulation \cite{aufiero_extended_2013}. This extension could be 
employed for immediate removal of fission product gases (e.g., Xe, Kr), which 
have a strong negative impact on core lifetime and breeding efficiency. 
Thus, using the built-in Serpent 2 Monte Carlo code online reprocessing \& 
refueling material burnup routine would significantly speed up 
computer-intensive full-core depletion simulations.

%As it was pointed out, the uncertainties of the nuclear data and its impact 
%on 
%a major safety and kinetics need to be evaluated with taking into account 
%online reprocessing and refueling. The fuel motion has a large impact on 
%safety and kinetic parameters and should be carefully investigated. This 
%would 
%require development a multi-physics model of the \gls{MSR} with some advanced 
%multi-physics software such as Moltres \cite{lindsay_introduction_2018}.

Additional physical models for fission product extraction efficiency will 
enrich the capabilities of SaltProc.


